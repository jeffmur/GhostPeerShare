\section{Video Similarity}
\label{sec:Related Video Similarity}
Using content-based algorithms, Shan and Lee \cite{Shan2002-sfs} presented a series of optimal mapping algorithms to detect similarity between video frames of unequal sizes. In addition, Wu, Zhuang, and Pan \cite{Wu2000-nf} present a similar shot graphing algorithm to identify similar videos within a large database of videos. An enhancement of Bhattacharyya was proposed by Loza, Mihaylova, Canagarajah, and Bull \cite{Loza2006-sso} to generate a particle filter that would iterate over each sample to predict and resample based on the similarity. Toward Privacy-Preserving Photo Sharing \cite{Ra2013-p3} presents an innovative photo encoding algorithm that enables the comparison of photos without disclosing the content of the frames. The future work section suggests that this method may also apply to videos. Overall, these methods may need significant changes to be compatible with fully homomorphic encryption, in that, core algorithms would need to be simplified to use basic arithmetic and may require many parameters to produce a score.

As a part of Similarity of Simultaneous Observation (SSO) \cite{Wu2019-SSO} and Proof of Presence Share (Pop-Share) \cite{Lagesse2021-PopShare}, a video was represented in probability distribution form, to compare two probability distributions, there are many methods to produce a similarity score. In this section, we cover alternative considerations for computing a similarity score, as well as, acknowledge alternative approaches to comparing videos.

Jensen-Shannon Divergence \cite{Lin1991-jsd}, represented as $JSD$ in Equation \ref{eq:jsd}, is a symmetric adaption of Kullback-Leibler Divergence that quantifies the similarity between two probability distributions. It is bounding the upper limit of KLD to 1, where 0 indicates high similarity and 1 indicates the maximum dissimilarity between the distributions.

\begin{equation}
    JSD(P || Q) = \frac{1}{2} D_{KL}(P || M) + \frac{1}{2} D_{KL}(Q || M)
    \label{eq:jsd}
\end{equation}

Pearson Correlation Coefficient \cite{Pearson1896-hp}, represented as $PCC$ in Equation \ref{eq:pearson}, measures the linear correlation between two variables, producing a value between -1 and 1. This coefficient assumes a linear relationship and may not capture nonlinear associations, which could lead to misleading conclusions in certain contexts. 

\begin{equation}
    r = \frac{\sum_{i=1}^{n} (X_i - \bar{X})(Y_i - \bar{Y})}{\sqrt{\sum_{i=1}^{n} (X_i - \bar{X})^2} \sqrt{\sum_{i=1}^{n} (Y_i - \bar{Y})^2}}
    \label{eq:pearson}
\end{equation}


Dynamic Time Warping (DTW) \cite{Sakoe1978-dtw}, represented as $C(i,j)$ in Equation \ref{eq:dtw}, is an algorithm that measures similarity between two-time series by exhaustively finding the optimal alignment between them, even if they have different speeds or lengths. However, DTW has a high computational cost, especially for long sequences, which can limit its use in practice. 

\begin{equation}
    C(i, j) = D(i, j) + \min \begin{cases}
        C(i-1, j) \\
        C(i, j-1) \\
        C(i-1, j-1)
    \end{cases}
    \label{eq:dtw}
\end{equation}

