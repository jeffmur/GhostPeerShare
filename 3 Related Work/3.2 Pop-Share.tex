\section{Proof of Presence Share}
\label{sec:Related Pop-Share}
Proof of Presence Share (Pop-Share) \cite{Lagesse2021-PopShare} introduces a novel approach for accurately detecting similar video scenes with strong privacy guarantees. Pop-Share is based on the pre-processing methodology of Similarity of Simultaneous Observation (SSO) \cite{Wu2019-SSO} to convert videos into byte-count arrays and adapts the distance measure algorithms from SSO to be compatible with Fully Homomorphic Encryption (FHE).

Developed by Wu and Lagesse, SSO was designed to detect streaming Wi-Fi cameras by comparing the probability distribution between network traffic and recording videos on the network. In addition, SSO contributes a computationally efficient way to transform video frames into probability distribution form to be compared by various statistical measures, including Jensen-Shannon Divergence (JSD), Kullback-Leibler Divergence, and Dynamic Time Warping (DTW) \cite{Sakoe1978-dtw}. The pre-processing approach of SSO slices the video into one-second segments, computes the total number of the bytes for all frames in each segment, and normalizes the byte count array so that it summates to one. Pop-Share could not adapt JSD and DTW to be compatible with FHE, so Cramer Distance and the Bhattacharyya Coefficient were selected to handle encrypted distance measure computations.

Pop-Share depended on SEAL 3.3 to handle fully homomorphic operations. Once both videos were preprocessed, the transformation was irreversible. In addition, the arrays were encrypted and computed using the plaintext counterpart, resulting in less noise accumulation than when using only ciphertext. In a distributed peer-to-peer application, the initiator shares an encrypted video for the recipient to apply their plaintext array onto the untrustworthy ciphertext. The modified ciphertext was returned to the originator to be decrypted, and the summation of the decrypted array represents the corresponding similarity score. This was performed for each similarity score measure. A notable limitation of this approach is that Pop-Share's implementation was tightly coupled with a static version of SEAL, making the results difficult to reproduce.

This method demonstrated that SSO generates a unique, one-way signature of videos that can be used to compare for similarity without access to the video's content. For this project, the results from Pop-Share serve as a baseline comparator for the modular Flutter re-implementation of the Pop-Share peer-to-peer application.
