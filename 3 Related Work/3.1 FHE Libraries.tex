\section{Fully Homomorphic Encryption Libraries}

Simple Encrypted Arithmetic Library (SEAL) [1] was released in November 2015 by Microsoft Research to facilitate the use of homomorphic encryption, making it easier for developers to implement encryption schemes like Brakerski/Fan-Vercauteren (BFV) [2] , Brakerski-Gentry-Vaikuntanathan (BGV) [3] [4], and Cheon-Kim-Kim-Song (CKKS) [5]. While Microsoft SEAL offers a powerful tool for homomorphic encryption, it has seen limited adoption in the mobile development community due to its complexity and the need for significant domain knowledge and manual compilation. This project addresses these limitations by introducing an interface designed to streamline the integration process and make SEAL more accessible as one of the supported backend libraries. This simplifies the use of homomorphic encryption in mobile applications, bridging the gap between advanced encryption techniques and practical, real-world deployment.

Open-Source Fully Homomorphic Encryption Library (OpenFHE) [6] was released in July 2022 by the open-source community as the successor to previous homomorphic encryption libraries, PALISADE [7] and HElib [8]. OpenFHE is designed with a modular architecture, allowing support for more homomorphic encryption schemes than Microsoft SEAL. However, as a relatively new tool, OpenFHE faces limitations: its recent release means it is still evolving, and it has not yet undergone a comprehensive peer-reviewed benchmark comparison against more established libraries like Microsoft SEAL.

Both SEAL and OpenFHE are written in C++, which provides significant performance advantages due to the language’s efficiency and low-level memory management capabilities. This allows for optimized execution of homomorphic encryption operations, making them suitable for resource-intensive applications. However, not all applications are written in C++, requiring an interface or wrapper to facilitate integration with other programming languages. This additional layer enables developers to leverage the performance benefits of compiled C code, while maintaining compatibility with applications written in other languages.
PySEAL [9] presents a wrapper around a static, compiled version of Microsoft SEAL, allowing Python applications to access homomorphic encryption functionalities. This approach forwards all lower-level C++ classes and methods, enabling developers to utilize SEAL's capabilities within a Python environment. However, this method has limitations, primarily the need for ongoing maintenance to keep the interface aligned with the latest SEAL updates. Breaking changes introduced in new versions can complicate version control, making it challenging for developers to maintain compatibility.

Python for Homomorphic Encryption Libraries (Pyfhel) [10] offers a robust solution by defining C abstraction methods to translate C++ classes into atomic operations. This abstraction layer modularizes the backend libraries, allowing for support of multiple homomorphic encryption libraries, including SEAL and PALISADE. While this approach is more flexible and supports a drop-in replacement of backend libraries, it contains additional complexity for maintenance of the abstraction layer and version control of backend libraries. This project adopts the methodology used in Pyfhel to implement a similar abstraction in Dart, aiming to provide an efficient and intuitive interface for leveraging homomorphic encryption.
