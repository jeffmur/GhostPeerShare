\section{Fully Homomorphic Encryption Libraries}

Open-Source Fully Homomorphic Encryption Library, OpenFHE [8], is a state-of-the-art C++ library born from a merger of PALISADE, HElib and HEAAN. OpenFHE supports modern schemes and can be configured to use hardware acceleration.

From existing systems, OpenFHE adapted the modular design of PALISADE [9] with the built-in support of BGV, BFV, CKKS and FHEW schemes. From HElib [10] an efficient homomorphic encryption library primarily focused on BGV and CKKS schemes, pioneered research and early development in this domain. HEAAN [11] was integrated to support arithmetic operations on approximate numbers while other schemes only support integer based arithmetic.

Compared to SEAL, OpenFHE library has an actively growing community with modular integrations of existing systems. The documentation and development guides for OpenFHE are much more modern and comprehensive. However, SEAL is much more established within the research community. Regarding performance, there have not been any peer reviewed publications directly comparing the performance of Microsoft SEAL and OpenFHE.

Python for Homomorphic Encryption Libraries (Pyfhel) [2] offers a robust solution by defining C abstraction methods to translate C++ classes into atomic operations. This abstraction layer modularizes the backend libraries, allowing for support of multiple homomorphic encryption libraries, including SEAL and PALISADE. While this approach is more flexible and supports a drop-in replacement of backend libraries, it contains additional complexity for maintenance of the abstraction layer and version control of backend libraries. This project adopts the methodology used in Pyfhel to implement a similar abstraction in Dart, aiming to provide an efficient and intuitive interface for leveraging homomorphic encryption.
