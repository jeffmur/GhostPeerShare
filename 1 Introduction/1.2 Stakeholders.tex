\section{Stakeholders}
\label{sec:Stakeholders}
Microsoft pioneered the Fully Homomorphic Encryption (FHE) research and led the development of the Simple Encryption Arithmetic Library (SEAL) in 2018. SEAL established a foundation for efficient homomorphic encryption, which made it accessible to the open-source community of security researchers and developers. SEAL was among the first of many libraries within the homomorphic encryption community. Other dominant libraries included the Homomorphic Encryption Library (HElib) \cite{HElib} and Privacy-Preserving Approximate Secure Computation for Encrypted Data (PALISADE) \cite{PALISADE}. Collectively, the open-source community aimed to combine their efforts into the Open-Source Fully Homomorphic Encryption Library (OpenFHE). A flexible, community-driven project that was initially released in 2022. Compared to SEAL, OpenFHE offers support for additional encryption schemes for evaluating Boolean circuits and arbitrary functions over larger plaintext spaces using lookup tables. However, OpenFHE is a relatively new framework with a growing community but requires further adoption in the research community to assess its proficiency. Currently, developers and security researchers are limited in creating FHE applications within the programming language constraints of C or C++, as a result, the mobile community must overcome significant hurdles to access this functionality.

Flutter, a state-of-the-art software development kit by Google captured 46\% of the cross-platform application market \cite{vailshery-flutter-statista}. This framework is built on top of Dart, a programming language recognized for its high performance and robust plugin system. Together, they provide compatibility with all major operating systems, specifically Android, Linux, macOS, iOS, and Windows. As a beneficiary of this work, the growing Flutter community may integrate Fully Homomorphic Encryption (FHE) into existing or new applications. This framework enables security researchers and developers to rapidly develop applications without the required prerequisite knowledge of the underlying C FHE libraries.
