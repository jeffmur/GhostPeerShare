\section{Contributions}
The Pyfhel library [2] provides a native Python interface to SEAL cryptosystems, abstracting core SEAL functionalities for easier use. While Pyfhel supports cross-platform compatibility on Windows, Linux, and macOS, it currently lacks support for mobile platforms like Android and iOS. This project adopts a similar approach, creating a modular interface that supports multiple backend libraries and manages them as Dart plugins.

A key contribution of this work is its modular design, which enables the seamless integration and versioning of different backend libraries through Dart plugins. With the use of CMake configurations, each C library can be synchronized and recompiled as new versions become available, simplifying the update process. Automating these compilations reduces maintenance overhead, ensuring that the libraries remain up-to-date with minimal manual intervention.

To assess both the performance and accuracy of the library, we re-implement the Proof of Presence Share [3] methodology. This method calculates similarity scores between videos without exposing their content, providing strong privacy guarantees. We set up cameras from various angles recording simultaneously and compare the videos for similarity, as well as, perform comparisons against other scenes of the same duration. Using a supervised artificial neural network, we classify the distance measures of Kullback-Leibler Divergence [4], Bhattacharyya Coefficient [5], and Cramer Distance [6] for training and testing so that the model can differentiate between scenes that are similar and those that are distinct. This classification enables us to accurately evaluate the library's effectiveness in identifying matching scenes while maintaining privacy. By analyzing the model’s accuracy across various similarity scores, we gain insight into the precision and reliability of our implementation in distinguishing matching scenes without revealing the videos’ content.