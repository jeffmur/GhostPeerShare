\section{Contributions}
The Pyfhel library \cite{Ibarrondo2021-Pyfhel} provides a native Python interface to SEAL cryptosystems, abstracting core SEAL functionalities into a more generic interface. While Pyfhel supports cross-platform compatibility on Windows, Linux, and macOS, it currently lacks support for mobile platforms, e.g. Android and iOS. This project adopts a similar approach, creating a modular interface that supports multiple backend libraries.

A key contribution of this work is its modular design, which enables the seamless integration and versioning of different backend libraries as Dart plugins. With the use of CMake configurations, each C library can be synchronized and recompiled as new versions become available, simplifying the update process. Automating these compilations reduces maintenance overhead, ensuring that the libraries remain up-to-date with minimal manual intervention.

To assess the library's performance and accuracy, we re-implement the Proof of Presence Share \cite{Lagesse2021-PopShare} methodology. This method calculates similarity scores between videos without exposing their content. We set up cameras from various angles to record simultaneously and compare the videos for similarity, as well as, perform comparisons against other scenes of the same duration. Using a supervised artificial neural network, we classify the distance measures of Kullback-Leibler Divergence \cite{Kullback1951-bg}, Bhattacharyya Coefficient \cite{Bhattacharyya1933-fw}, and Cramer Distance \cite{Cramer1928-sw} for training and testing. The binary classification model is trained to differentiate between pairs of scenes, classifying them as either similar or different based on the unique representations generated by our system. By analyzing the model’s accuracy through many predictions, we gain insight into the precision and reliability of our implementation in distinguishing matching scenes without revealing the videos’ content.
