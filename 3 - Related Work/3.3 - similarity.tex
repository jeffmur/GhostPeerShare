\section{Video Similarity}

As a part of SSO and Pop-Share, a video was represented in probability distribution form, in order to compare two probability distributions, there are many methods to produce a similarity score. In this section, we cover all considerations for generating a similarity score, as well as, acknowledge alternative approaches to comparing videos.

Cramer’s Distance (CD) [15] is a measure used to quantify the dissimilarity between two probability distributions. It is based on the concept of the characteristic function and finds applications in statistical inference and hypothesis testing. One limitation of Cramer’s Distance is its sensitivity to sample size, which can result in inaccuracies, particularly when the sample size is small.
Bhattacharyya Coefficient (BC) [16] is a measure that quantifies the amount of overlap between two probability distributions, making it useful in various fields such as pattern recognition and image processing. The coefficient ranges from 0 to 1, where a value closer to 1 indicates higher similarity between the distributions.

Kullback-Leibler Divergence (KLD) [17] is a widely used measure of how one probability distribution diverges from a second, expected probability distribution. A limitation of KLD is that it is not symmetric, meaning that the divergence from distribution P to Q is not necessarily the same as from Q to P, which may complicate its interpretation.
Jensen-Shannon Divergence (JSD) [18] is a symmetric version of KLD that quantifies the similarity between two probability distributions. It is widely used in information retrieval and clustering. A limitation is that while it provides a measure of similarity, it may not capture finer details of the distributions being compared.

The Pearson Correlation Coefficient (PCC) [19] measures the linear correlation between two variables, producing a value between -1 and 1. It is a fundamental statistical tool in various fields, including psychology and economics. However, this coefficient assumes a linear relationship and may not capture non-linear associations, which could lead to misleading conclusions in certain contexts. 
Dynamic Time Warping (DTW) [20] is an algorithm used to measure similarity between two temporal sequences that may vary in speed. It has applications in speech recognition, data mining, and video analysis. One limitation of DTW is its computational complexity, particularly for long sequences, which can make it less suitable for real-time applications. 
Alternatives to SSO are advanced applications on top of existing similarity score algorithms. These novel approaches are typically not restricted to resource-constrained environments.

Using content-based algorithms, Shan and Lee [21] presented a series of optimal mapping algorithms to detect similarity between video frames of unequal sizes. In addition, Wu, Zhuang, and Pan [22] present a similar shot graphing algorithm with the intention of identifying similar videos within a large database of videos. An enhancement of Bhattacharyya, was proposed by Loza, Mihaylova, Canagarajah, and Bull [23] to generate a particle filter that would iterate over each sample to predict and resample based on the similarity. Overall, these methods may be adapted to be implemented with fully homomorphic encryption, however they would need to be simplified to use basic arithmetic, and may require many parameters to produce a score.
