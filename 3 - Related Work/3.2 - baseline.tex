\section{Proof of Presence Share}

Proof of Presence Share (Pop-Share) [11] introduces a novel approach for accurately detecting similar video scenes with strong privacy guarantees, presenting two applications of fully homomorphic encryption deployed within a client-server architecture and a distributed peer-to-peer network. In order to accurately detect similar video scenes, Pop-Share proposed a new application and evaluation of Similarity of Simultaneous Observation (SSO) [12]. 

Developed by Wu and Lagesse, SSO was intended to detect streaming Wi-Fi cameras by comparing the probability distribution between network traffic and recording videos on the network. In addition, SSO contributed a computationally efficient way to transform video frames into probability distribution form to be compared by various statistical measures including JSD, KLD, and DTW. The preprocessing approach of SSO slices the video into one second segments, computes the total number of the bytes for all frames in each segment, and normalizes the byte count array, so that it summates to one. However, JSD and DTW could not be simplified enough to be used in fully homomorphic operations, so Pop-Share leveraged Cramer Distance and Bhattacharyya Coefficient to handle the encrypted similarity score measures.

Pop-Share was developed in 2020, their implementation depended on SEAL 3.3 to handle fully homomorphic operations. Once both videos were preprocessed, the transformation was irreversible, having strong privacy guarantees. In addition, the arrays were encrypted and computed using the plaintext counterpart, resulting in less noise accumulation than when using only ciphertext. In a distributed peer-to-peer application, the initiator shares an encrypted video for the recipient to apply their plaintext array onto the untrustworthy ciphertext. The modified ciphertext was returned to the originator to be decrypted and the summation of the decrypted array represents the corresponding similarity score. This was performed for each similarity score measure. A notable limitation of this approach is that their implementation was tightly coupled with a static version of SEAL, making the results difficult to reproduce.

This method demonstrated that SSO generates a unique, one-way signature of videos that can be used to compare for similarity without access to the content of the video. For this project, the results from Pop-Share serve as a baseline comparator for the modular Flutter re-implementation of the Pop-Share peer to peer application.

An alternative approach focuses on using computer vision to extract privacy-sensitive video objects [13] instead of summarizing video frames per segment. This advanced method is particularly suited for video databases hosted in the cloud, as it requires substantial computational resources for feature extraction and similarity score calculations.

Toward Privacy-Preserving Photo Sharing (P3) [14] presents an innovative photo encoding algorithm that enables the comparison of photos without disclosing the content of the frames. The future work section suggests that this method may also be applicable to videos.
