\section{Distance Measure Plugin}

This Dart plugin calculates three distance measures: Kullback-Leibler Divergence, Cramer Distance, and Bhattacharyya Coefficient, as well as their Fully Homomorphic Encryption (FHE) counterpart. To manage the complexity of these computations, the plugin only supports operations on lists of plaintext doubles, which are then encrypted for processing. In this section, we explain the fully homomorphic computations for each algorithm, along with details on testing and distribution.

%TODO%
Fully homomorphic encryption (FHE) enables computations on encrypted data, ensuring that data privacy is maintained throughout the process. Each algorithm’s encrypted score remains interpretable because it retains the mathematical relationships needed for distance calculation without exposing the underlying plaintext values. For more information on the equations of the plaintext scores, refer to Section 2.3. However, a major limitation in FHE schemes is their inability to directly support division due to the mathematical structures in FHE, which operate within polynomial or modular arithmetic. Workarounds, such as using multiplicative inverses, do exist, though they add complexity and are unreliable.

To accommodate the limitations of FHE, we cannot directly encrypt the input array of doubles for score computation. Instead, we adjust certain parameters tailored to each algorithm’s requirements before encryption. The encrypted parameters and any associated data are stored as binary files, with each algorithm (e.g., $kld\_logX$) using 60 binary files, each representing one second of a one-minute video. This structure is particularly helpful for cases where video lengths differ, as it simplifies the trimming process. The binary files are then archived into an .enc file format, which is unique to our application and contains both the encrypted data and a metadata JSON file with relevant video details (e.g., timestamp, length, frames per second). The archive averages around 2 MB on Linux for 1 minute of video.

To compute the encrypted Kullback-Leibler Divergence using FHE, noted as $KLD_{enc}$ in Equation \ref{eq:fhe_kld_enc}, the ciphertext modification requires three parameters: $X$, $logX$, and $Y$, where $X$ and $logX$ are encrypted arrays, and $Y$ is plaintext. The modified ciphertext is obtained by multiplying each encrypted double in $X$ by the difference between $logX$ and $logY$. This yields an encrypted list of doubles, which, when decrypted and summed, results in the divergence of $Y$ from $X$, shown in Equation \ref{eq:fhe_kld_decrypt}.

\begin{equation}
        KLD_{enc} = ( \log(X)_{enc}^i - \log(Y_{plain}^i) ) \cdot X_{enc}^i
    \label{eq:fhe_kld_enc}
\end{equation}

\begin{equation}
        KLD_{plain} = \sum_{i=1}^{n} D_K(KLD_{enc})
    \label{eq:fhe_kld_decrypt}
\end{equation}

For the Bhattacharyya Coefficient, indicated as $BC_{enc}$ in Equation \ref{eq:fhe_bc_enc}, the square root of $X$ is taken before encryption. This calculation involves two arrays, $sqrtX$ and $Y$, both floating-point arrays. We multiply each encrypted double in $sqrtX$ by the square root of $Y$, yielding an encrypted list. Decrypting and summing this list gives the measure of overlap between the two probability distributions, shown in Equation \ref{eq:fhe_bc_decrypt}.

\begin{equation}
    BC_{enc} = (\sqrt{X})_{enc}^i \cdot \sqrt{Y_{plain}^i}
    \label{eq:fhe_bc_enc}
\end{equation}

\begin{equation}
    BC_{plain} = \sum_{i=1}^{n} D_K(CD_{enc})
    \label{eq:fhe_bc_decrypt}
    \title{Foo}
\end{equation}

To calculate the Cramer Distance, or $CD_enc$ in Equation \ref{eq:fhe_cd_enc}, the cumulative distribution of $X$ is computed iteratively so that the last element sums to one. This requires two arrays, $X$ and $Y$, of floating-point numbers. The computation squares the difference between corresponding elements in $X$ and $Y$, resulting in an encrypted list of values, which, when decrypted, summed, and square-rooted, provides the distance between the distributions, shown in Equation \ref{eq:fhe_cd_decrypt}.

\begin{equation}
    CD_{enc} = (X_{i} - Y_{i})^2
    \label{eq:fhe_cd_enc}
\end{equation}

\begin{equation}
    CD_{plain} = \sqrt{\sum_{i=1}^{n} D_K(CD_{enc})}
    \label{eq:fhe_cd_decrypt}
\end{equation}

The plugin undergoes automated testing via GitHub Actions, which validates scores generated from identical input data to within a 7-decimal-point precision (as shown in Table \ref{table:mean-error}, which details the noise introduced by FHE). Testing covers both the symmetric and asymmetric behavior of each algorithm. A core design goal of this library is modularity, allowing for the future integration of additional distance measures using FHE. This work introduces the FHEL Dart plugin, creating a foundational Dart plugin for distance measurements that supports fully homomorphic encryption.
