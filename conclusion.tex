GhostPeerShare significantly advances the accessibility of Fully Homomorphic Encryption (FHE) within the mobile community. Key contributions demonstrate the efficiency of the Fully Homomorphic Encryption Library (FHEL) and ease of integration into other Dart plugins or Flutter applications.
To measure the efficiency of our application, we benchmark GhostPeerShare against Proof of Presence Share (Pop-Share).
GhostPeerShare generated the byte-count array 7 minutes slower than Pop-Share when pre-processing videos. However, our implementation may be improved using multi-threading or replaced with the same underlying FFmpeg video processing library as Pop-Share.
When trained on a dataset with continuous movement and tested on a dataset with infrequent movement, the application achieved 0.9614 precision and an F1 score of 0.9709. These results indicate that GhostPeerShare consistently generated a distinct representation of each video in the dataset.
When executing FHE operations, GhostPeerShare demonstrated similar performance to Pop-Share. The computational times of FHE distance measure under CKKS reveal that our implementation was within 10\% of Pop-Share. These results suggest that the FHEL plugin maintains Microsoft SEAL's high performance.
Overall, GhostPeerShare re-implements Pop-Share with efficient FHE computation. However, the application requires significant improvement in the pre-processing implementation.

The scope of this project was to develop an Android and Linux application. The FHEL plugin requires additional Conan profiles to support additional operating systems, including Windows, iOS, and macOS. However, this involves prerequisite knowledge in CMake to extend our implementation and depends on the compatibility of the underlying C++ library. The binary distribution system can also be extended with the automated GitHub actions run on macOS and Windows runners to compile their respective dynamic libraries for release. 

A significant limitation of the FHEL plugin is its versioning strategy. Currently, the plugin supports Microsoft SEAL; however, when extended to support additional libraries, changes to any backend library require a minor version change. Rather than deploying a monolith plugin, the version system should deploy multiple artifacts from the same shared codebase, for example, $fhel\_seal$ and $fhel\_openfhe$. The proposed version system allows each FHEL plugin to align its version numbers with the underlying libraries' versions, simplifying version management for package maintainers and consumers.

Ultimately, this project lays the foundation for future applications of FHE in mobile environments, especially in fields like healthcare and secure data transactions, where privacy and accuracy are paramount. By making SEAL accessible within a mobile-compatible, modular framework, we are opening pathways for broader adoption of FHE and setting the stage for innovation in mobile security.
