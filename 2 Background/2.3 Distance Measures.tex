\section{Distance Measures}
\label{sec:Background Distance Measures}
To accurately measure the similarity between the two videos, this project implements the Kullback-Leibler Divergence, Bhattacharyya Coefficient, and Cramer’s Distance. Pop-Share \cite{Lagesse2021-PopShare} established these metrics as an effective indicator for comparing video similarity. Unlike other probability distribution algorithms that involve complex calculations, these three metrics rely primarily on element-wise addition, subtraction, and multiplication of vectors. This simplicity is crucial for their integration with Fully Homomorphic Encryption, as it allows for efficient computation on a list of encrypted floating-point numbers.

Kullback-Leibler Divergence \cite{Kullback1951-bg}, represented as $KLD$ in Equation \ref{eq:kld}, is a state-of-the-art measure of how one probability distribution diverges from a second, expected probability distribution. A limitation of KLD is that it is not symmetric, in that the divergence from distribution P to Q is not necessarily the same as from Q to P, which may complicate its interpretation. The divergence ranges from 0 to positive infinity, where a value closer to 0 indicates high similarity between the distributions.

\begin{equation}
    KLD = \sum_{i} P(i) \log \left( \frac{P(i)}{Q(i)} \right)
    \label{eq:kld}
\end{equation}

Bhattacharyya Coefficient \cite{Bhattacharyya1933-fw}, represented as $BC$ in Equation \ref{eq:bc}, is a measure that quantifies the amount of overlap between two probability distributions. The coefficient ranges from 0 to 1, where a value closer to 1 indicates high similarity between the distributions.

\begin{equation}
    BC(P, Q) = \sum_{x} \sqrt{P(x) Q(x)}
    \label{eq:bc}
\end{equation}

Cramer’s Distance \cite{Cramer1928-sw}, represented as $CD$ in Equation \ref{eq:cd}, is a measure used to quantify the dissimilarity between two probability distributions. It is sensitive to small sample sizes, which can result in inaccuracies. The distance ranges from 0 to 1, where a value closer to 0 indicates high similarity between the distributions.

\begin{equation}
    CD = \sqrt{\sum_{i} (P_{i} - Q_{i})^2}
    \label{eq:cd}
\end{equation}
