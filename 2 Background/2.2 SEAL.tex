\section{Simple Encrypted Arithmetic Library}
\label{sec:Background SEAL}
To perform computations on encrypted data, this project utilizes Simple Encrypted Arithmetic Library (SEAL) \cite{sealcrypto}, an open-source homomorphic encryption library developed by Microsoft. The library supports limited arithmetic including addition, subtraction, and multiplication to be performed directly on encrypted data without needing to decrypt it. We chose SEAL primarily for its maturity, prevalent adoption in research, and use in Proof of Presence Share \cite{Lagesse2021-PopShare}.

SEAL supports three Fully Homomorphic Encryption (FHE) schemes: Brakerski-Fan-Vercauteren (BFV) \cite{fan2012-bfv}, Brakerski-Gentry-Vaikuntanathan (BGV) \cite{brakerski2012-bgv}, and Cheon-Kim-Kim-Song (CKKS) \cite{Cheon2017-CKKS}. BFV and BGV are integer-based FHE schemes that allow for computations on encrypted integers but differ in their ciphertext structure and noise management techniques. BFV encodes the message with the most significant bits of the ciphertext and accumulates more noise growth for each multiplication circuit. When too much noise accumulates, the ciphertext cannot be decrypted. For simpler computations, BFV offers better performance. BGV encodes the message in the least significant bits and manages noise through modulus switching. This difference in noise management makes BGV suitable for deeper computations with many sequential operations. 

CKKS is specifically designed for scenarios where approximate arithmetic is acceptable, and this trade-off allows for more efficient computations on encrypted real numbers. This scheme is efficient in performing computations, by representing real numbers as complex numbers. The encoding allows for efficient arithmetic operations, specifically addition, multiplication, and subtraction. When decoding, the result is transformed into a meaningful representation of the true value, with some degree of error. 
