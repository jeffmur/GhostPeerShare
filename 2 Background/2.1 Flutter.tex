\section{Flutter}
Flutter is a modern, open-source software development kit designed to build stylish and efficient applications for mobile, web, and desktop within a single code base. Flutter is built on top of Dart; an object-oriented language with C-style syntax. Dart includes advanced features out of the box such as garbage collection for memory management, synchronous and asynchronous handlers, null safety, and compile-time data type checking. We selected Flutter for its low development cost and wide variety of plugins.

The plugin at the root of this project's design, the Foreign Function Interface allows the application to invoke functions from pre-compiled C libraries. Each target platform, specifically Windows, Linux, Android, iOS, and macOS requires different methods to compile, however, the fundamental method of invoking C functions and working with native data structures remains the same across platforms. The plugin is limited to transforming primitive data types, e.g. Booleans, Integers, and Characters, restricting the parameters of native C functions to primitive data types. However, a notable workaround to this problem is passing complex objects by memory address, known as pass by reference, for the underlying C library to access the object stored in this stack. 

To facilitate the networking between peer-to-peer Android devices, QuickShare \cite{samsung_quick_2020} was developed by Samsung as a file-sharing utility application for nearby wireless devices. It leverages Bluetooth to discover nearby wireless devices and optionally uses WiFi to transfer large files between two nodes. QuickShare is accessible through a Flutter plugin that enables users to share data using their device's native sharing capabilities. This allows users to easily send content to other apps, such as social media platforms, messaging apps, or email.

Alternatives such as React Native (with JavaScript) or native development languages with Kotlin (Android) and Swift (iOS) involve different compilation methods. React Native is a popular framework for cross-platform development, comparable to Flutter in plugins and community activity, however, lacks the native support of C interoperability and performance. React Native leverages Kotlin and Swift to configure and compile Android and iOS applications, respectively. This approach carries inherent challenges to developing and testing an application, often requiring additional dependencies to handle trivial maintenance. Flutter aims to streamline the process of building cross-platform applications by unifying the codebase, eliminating the need for separate code for each platform. 
