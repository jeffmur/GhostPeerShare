\section{Noise Accumulation}

When performing fully homomorphic operations, a small amount of noise is introduced that affects the accuracy of the decrypted distance measure. If too much noise is accumulated, the plaintext cannot be recovered. Table \ref{table:mean-error} represents the baseline Mean Error from Pop-Share, compared to our implementation. For Kullback-Leibler Divergence, our implementation introduced twice the amount of noise than the original. For Cramer Distance, our implementation introduced half of the amount of noise than the original. For Bhattacharyya Coefficient, our implementation introduced half of the amount of noise than the original.

% \usepackage{amsmath}
% \usepackage{graphicx}
% \usepackage{array}

\begin{table}[h!]
\centering
\begin{tabular}{|c|c|c|}
\hline
\textbf{Function} & \textbf{Pop-Share} & \textbf{GhostPeerShare} \\
\hline
KLD     & $4 \times 10^{-9}$ & $1.78 \times 10^{-6}$ \\
Cramer  & $8 \times 10^{-4}$ & $6.54 \times 10^{-10}$ \\
BC      & $1.3 \times 10^{-1}$ & $6.34 \times 10^{-10}$ \\
\hline
\end{tabular}
\caption{Comparison of Mean Errors for Different Functions}
\label{table:mean-error}
\end{table}


The variations in noise across different distance measures in this fully homomorphic encryption (FHE) implementation may be attributed to several factors. Advances in the SEAL library v3.3 from the original paper while our implementation uses v4.1, which underpins this work, have likely introduced optimizations that improve noise efficiency for certain operations, benefiting measures like Cramer Distance and Bhattacharyya Coefficient by reducing overall noise accumulation. Conversely, Kullback-Leibler Divergence (KLD) appears to have a more noise-sensitive implementation, possibly due to its reliance on logarithmic functions that amplify noise in homomorphic contexts. Table \ref{table:mean-error} highlights this distinction, suggesting that KLD’s unique computation structure makes it more susceptible to noise. Additionally, certain distance measures naturally require different types of arithmetic operations, impacting noise levels differently. KLD, for instance, might involve more complex calculations than Cramer Distance and Bhattacharyya Coefficient, leading to faster noise accumulation. Finally, the parameter choices in the encryption scheme, such as polynomial and ciphertext modulus, could also influence noise resilience, especially if these parameters are less optimized for KLD, contributing to the observed differences in noise behavior.
