\section{Noise Accumulation}

When performing fully homomorphic operations, a small amount of noise is introduced that affects the accuracy of the decrypted distance measure. If too much noise is accumulated, the plaintext cannot be recovered. Table \ref{table:mean-error} represents the baseline Mean Error from Pop-Share, compared to our implementation. For Kullback-Leibler Divergence, Cramer Distance, and Bhattacharyya Coefficient our implementation introduced half the amount of noise than the original.

\begin{table}[h!]
\centering
\begin{tabular}{|c|c|c|}
\hline
\textbf{Function} & \textbf{Pop-Share} & \textbf{GhostPeerShare} \\
\hline
KLD     & $4 \times 10^{-9}$ & $1.27 \times 10^{-11}$ \\
Cramer  & $8 \times 10^{-4}$ & $1.37 \times 10^{-9}$ \\
BC      & $1.3 \times 10^{-1}$ & $4.8 \times 10^{-10}$ \\
\hline
\end{tabular}
\caption{Comparison of Mean Errors for Different Functions}
\label{table:mean-error}
\end{table}


As illustrated in Figure \ref{fig:pop-share-distance-measure}, our Dart implementation exhibits less variance compared to the Python baseline, allowing for a clearer distinction in mean error scores. This reduced variance indicates a more stable performance in our implementation, reinforcing the reliability of the distance measures employed. Furthermore, through automated unit tests, we have ensured that the scores do not differ more than $1.0e-7$ epsilon, confirming the consistency and accuracy of our calculations.

Finally, the parameter choices within the encryption scheme-such as polynomial modulus degree, encoder scalar or array of modulus sizes-also play a significant role in noise management.
