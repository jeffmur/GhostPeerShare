\section{Field of View}
\label{sec:Field of View}
Borrowing the methodology from Pop-Share \cite{Lagesse2021-PopShare}, we compare videos without revealing their content, labeling each row of similarity scores with a one or zero to train a supervised artificial neural network for classifying new scenes. The binary labeling system assigns a one when two scenes are identical; otherwise, a zero is assigned. In addition, we label the first comparison pair as one, to train the model to match similar videos as well. In Dart, we exhaustively compare the plaintext byte count arrays to generate a row of labels along with the Kullback-Leibler Divergence (KLD), Bhattacharyya Coefficient (BC), and Cramer Distance (CD) for comparison between each one-second segment of videos of the same length. Prior to comparison, both videos have been manually aligned and trimmed to ensure consistency.

\begin{table}[b!]
\centering
\begin{tabular}{lccccc}
\hline
\textbf{System} & \textbf{F1 (\%)} & \textbf{Precision (\%)} & \textbf{Recall (\%)} & \textbf{Accuracy (\%)} & \textbf{Error (\%)} \\
\hline
PoP-Share[2]   & 96.63 & 97.73 & 95.56 & 95.16 & 4.84 \\
Handheld[2]    & 97.97 & 99.32 & 96.67 & 98.00 & 2.00 \\
SSO[6]         & 96.13 & 92.56 & 100.00 & 96.30 & 3.70 \\
GhostPeerShare & 97.09 & 96.14 & 98.05 & 98.05 & 1.95 \\
\hline
\end{tabular}
\caption{Direct Comparison of SSO-based Systems}
\label{table:ann_system_compare}
\end{table}


As shown in Table \ref{table:ann_system_compare}, we contribute GhostPeerShare to the previous analysis conducted in Pop-Share and Similarity of Simultaneous Observation (SSO) \cite{Wu2019-SSO} by employing supervised learning with an Artificial Neural Network (ANN) to accurately classify a binary score, representing the same or different scene based on the three distance measures. We trained and tested the model on a small dataset of 100 minutes of raw videos, which included three twenty-minute videos of low movement featuring an individual sitting at a desk, denoted as \textit{office}, and two twenty-minute videos of high movement capturing an individual vacuuming his living room, denoted as \textit{vaccum} in Appendix Table \ref{table:appendix_ann}.

For a practical analysis, the ANN was trained on a high-movement scene and tested on a low-movement scene, achieving an accuracy and recall of 98.05\%, a precision of 96.14\%, and an F1 Score of 97.09\%. The Multi-Layer Perceptron (MLP) classifier is part of the scikit-learn \cite{scikit-learn} library. The MLP Classifier used the Limited-memory Broyden-Fletcher-Goldfarb-Shanno (lbfgs) solver to apply weights within the neural network. The model was trained with two hidden layers, each with 10 neurons, and used Rectified Linear Unit (relu) a popular activation function in deep learning. This model demonstrates that GhostPeerShare consistently generates a unique representation of video data, and the model can accurately predict the binary label given the three distance measure scores for a different dataset.
