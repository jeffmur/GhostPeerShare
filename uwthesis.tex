%  ========================================================================
%  Copyright (c) 1985 The University of Washington
%
%  Licensed under the Apache License, Version 2.0 (the "License");
%  you may not use this file except in compliance with the License.
%  You may obtain a copy of the License at
%
%      http://www.apache.org/licenses/LICENSE-2.0
%
%  Unless required by applicable law or agreed to in writing, software
%  distributed under the License is distributed on an "AS IS" BASIS,
%  WITHOUT WARRANTIES OR CONDITIONS OF ANY KIND, either express or implied.
%  See the License for the specific language governing permissions and
%  limitations under the License.
%  ========================================================================
%

% Documentation for University of Washington thesis LaTeX document class
% by Jim Fox
% fox@washington.edu
%
%    Revised 2020/02/24, added \caption()[]{} option.  No ToC.
%
%    Revised for version 2015/03/03 of uwthesis.cls
%    Revised, 2016/11/22, for cleanup of sample copyright and title pages
%
%    This document is contained in a single file ONLY because
%    I wanted to be able to distribute it easily.  A real thesis ought
%    to be contained on many files (e.g., one for each chapter, at least).
%
%    To help you identify the files and sections in this large file
%    I use the string '==========' to identify new files.
%
%    To help you ignore the unusual things I do with this sample document
%    I try to use the notation
%       
%    % --- sample stuff only -----
%    special stuff for my document, but you don't need it in your thesis
%    % --- end-of-sample-stuff ---


%    Printed in twoside style now that that's allowed
%
 
\documentclass [11pt, proquest] {uwthesis}[2020/02/24]
 
%
% The following line would print the thesis in a postscript font 

% \usepackage{natbib}
% \def\bibpreamble{\protect\addcontentsline{toc}{chapter}{Bibliography}}

\setcounter{tocdepth}{1}  % Print the chapter and sections to the toc
 

% ==========   Local defs and mods
%

% --- sample stuff only -----
% These format the sample code in this document

\usepackage{alltt}  % 
\newenvironment{demo}
  {\begin{alltt}\leftskip3em
     \def\\{\ttfamily\char`\\}%
     \def\{{\ttfamily\char`\{}%
     \def\}{\ttfamily\char`\}}}
  {\end{alltt}}
 
% metafont font.  If logo not available, use the second form
%
% \font\mffont=logosl10 scaled\magstep1
\let\mffont=\sf
% --- end-of-sample-stuff ---

% Table 5.1
\usepackage{amsmath}
\usepackage{graphicx}
\usepackage{array}

% BibTeX
\usepackage{url}

\begin{document}
 
% ==========   Preliminary pages
%
% ( revised 2012 for electronic submission )
%

\prelimpages
 
%
% ----- copyright and title pages
%
\Title{GhostPeerShare}
\Author{Jeffrey Murray Jr}
\Year{2024}
\Program{Cybersecurity Engineering}

\Chair{Brent Lagesse}{Dr.}{Computing \& Software Systems}
\Signature{Geethapriya Thamilarasu}
\Signature{Yang Peng}

\copyrightpage

\titlepage  

 
%
% ----- signature and quoteslip are gone
%

%
% ----- abstract
%


\setcounter{page}{-1}
\abstract{%
This paper presents advancements in the integration of Fully Homomorphic Encryption (FHE) within mobile applications through the development of the Fully Homomorphic Encryption Library (FHEL) Plugin and the Distance Measure Plugin for Flutter. By leveraging Microsoft SEAL, we demonstrate high performance and responsiveness for encrypted computations on mobile devices, addressing the need for privacy and security in mobile software development. Our results indicate high accuracy and consistency in deriving video similarity scores using Kullback-Leibler Divergence, Bhattacharyya Coefficient, and Cramer’s Distance, facilitating effective secure data exchange. This work opens pathways for future applications of FHE, particularly in critical areas such as medical devices, enabling secure interactions between mobile applications and encrypted data from cloud service providers. Ultimately, our findings highlight the potential of FHE to enhance user privacy and security in modern mobile applications, laying the groundwork for future innovations in this field.
}
 
%
% ----- contents & etc.
%
\tableofcontents
\listoffigures
\listoftables
 
%
% ----- glossary 
%
\chapter*{Glossary}      % starred form omits the `chapter x'
\addcontentsline{toc}{chapter}{Glossary}
\thispagestyle{plain}
%
\begin{glossary}
\item[Fully Homomorphic Encryption (FHE)] A type of encryption that allow computations over encrypted data without access to the decryption key.

\item[Cheon-Kim-Kim-Song (CKKS) Scheme] A fully homomorphic encryption scheme that supports approximate arithmetic on real numbers used for decimal-based computations.

\item[Simple Encrypted Arithmetic Library (SEAL)] An open-source library developed by Microsoft for performing fully homomorphic encryption arithmetic on encrypted data.

\item[Kullback-Leibler Divergence (KLD)] A measure of how one probability distribution diverges from another, ranges from 0 to positive infinity, where 0 represents identical probability distributions.

\item[Bhattacharyya Coefficient (BC)] A similarity measure that quantifies the amount of overlap between two probability distributions, ranges from 0 to 1, where 1 represents identical probability distributions.

\item[Cramer's Distance (CD)] A statistical measure used to quantify dissimilarity between two probability distributions, ranges from 0 to 1, where 0 represents identical probability distributions.

\item[Flutter] An open-source, cross-platform framework by Google for building natively compiled applications for mobile, web, and desktop from a single codebase.

\item[Dart] A programming language optimized for building fast applications on any platform, used in the development of Flutter apps.

\item[FFmpeg] A multimedia framework commonly used for video and audio processing, used here for video preprocessing in mobile applications.

\item[OpenCV] An open-source computer vision and machine learning software library that provides tools for image and video processing.

\item[GitHub Actions] A CI/CD tool that automates tasks such as building, testing, and deploying software, supporting the automation of plugin testing and deployment in this project.

\end{glossary}

 
%
% ----- acknowledgments
%
% \acknowledgments{% \vskip2pc
%   % {\narrower\noindent
%   The author wishes to express sincere appreciation to
%   University of Washington, where he has had the opportunity
%   to work with the \TeX\ formatting system,
%   and to the author of \TeX, Donald Knuth, {\it il miglior fabbro}.
%   % \par}
% }

%
% end of the preliminary pages

%
% ==========      Text pages
%

\textpages
 
% ========== Chapter 1
 
\chapter {Introduction}

This chapter outlines the problem, identifies the beneficiaries, and describes the project's contributions. In Section \ref{sec:Problem Statement}, we summarize the problem statement. In Section \ref{sec:Stakeholders}, we identify beneficiaries and major stakeholders within the Fully Homomorphic Encryption (FHE) domain. In Section \ref{sec:Contributions}, we present a modular interface that grants access to FHE functionality on a state-of-the-art software development platform.

\section{Problem Statement}
\label{sec:Problem Statement}
The Microsoft Simple Encrypted Arithmetic Library (SEAL) \cite{sealcrypto} requires significant pre-requisite knowledge to compile and embed the binary within resource-constrained devices. These prerequisites have limited the accessibility of SEAL, posing a significant obstacle to the adoption within the mobile community.

\section{Stakeholders}
Microsoft was a pioneer in the Fully Homomorphic Encryption research, leading the development of the Simple Encryption Arithmetic Library, SEAL in 2018. SEAL established a foundation for efficient homomorphic encryption, making it accessible for developers and researchers. The open-source community soon contributed to the field with OpenFHE, a flexible, community-driven project that was initially released in 2020. OpenFHE expands upon the capabilities of SEAL, offering a collaborative platform for researchers and developers interested in advancing encryption technology.

Flutter, a versatile and state-of-the-art framework by Google, now captures 46\% of the cross-platform framework market (Vailshery 2024). Recognized for its ability to support high-performance, cross-platform applications, Flutter includes a powerful plugin system with a Foreign Function Interface (FFI). This interface enables seamless integration with C libraries, offering null safety, managed memory, and compatibility across major operating systems, including Android, Linux, macOS, iOS, and Windows.

By leveraging Dart, a more portable programming language, developers and security researchers can interact directly with FHE backend libraries. This integration reduces the complexity of working with SEAL or OpenFHE, allowing developers to rapidly build cross-platform applications with homomorphic encryption capabilities. Furthermore, security researchers can extend the application to support additional C/C++ libraries, broadening the scope of FHE implementations and facilitating wider distribution to end-users.

\section{Contributions}
\label{sec:Contributions}
The Pyfhel library \cite{Ibarrondo2021-Pyfhel} provides a native Python interface to SEAL cryptosystems, abstracting core SEAL functionalities into a more generic interface. While Pyfhel supports cross-platform compatibility on Windows, Linux, and macOS, it currently lacks support for mobile platforms, e.g. Android and iOS. This project adopts a similar approach, creating a modular interface that supports multiple backend libraries.

This project contributes a modular, Dart plugin that abstracts and implements Microsoft Simple Encrypted Arithmetic Library. The abstraction design enables the integration and versioning of additional backend libraries. With the use of CMake configurations, each C library can be synchronized and recompiled as new versions become available, simplifying the update process. Automating these compilations reduces maintenance overhead, ensuring that the libraries remain up-to-date with minimal manual intervention.

To assess the library's performance and accuracy, we re-implement the Proof of Presence Share \cite{Lagesse2021-PopShare} methodology. This method calculates similarity scores between videos without exposing their content. We set up cameras from various angles to record simultaneously and compare the videos for similarity, as well as, perform comparisons against other scenes of the same duration. Using a supervised artificial neural network, we classify the distance measures of Kullback-Leibler Divergence \cite{Kullback1951-bg}, Bhattacharyya Coefficient \cite{Bhattacharyya1933-fw}, and Cramer Distance \cite{Cramer1928-sw} for training and testing. The binary classification model is trained to differentiate between pairs of scenes, classifying them as either similar or different based on the unique representations generated by our system. By analyzing the model’s accuracy through many predictions, we gain insight into the precision and reliability of our implementation in distinguishing matching scenes without revealing the videos’ content.


% ========== Chapter 2
 
\chapter{Background}

This chapter outlines the technologies and methods used in this research. In Section \ref{sec:Background Flutter}, we start with Flutter, an open-source framework for building cross-platform applications, and highlight its benefits for mobile development. In Section \ref{sec:Background SEAL}, we introduce the Simple Encrypted Arithmetic Library and explain the process of performing computations on encrypted data. In Section \ref{sec:Background Distance Measures}, we discuss the distance measures used for privacy-preserving video similarity analysis, focusing on Kullback-Leibler Divergence, Bhattacharyya Coefficient, and Cramer’s Distance. We examine the properties and uses of each metric to show their importance in our work.

\section{Flutter}
To support Android and Linux operating systems, Flutter is a modern, open-source framework designed to build natively compiled applications for mobile, web, and desktop using a single codebase. It is based on Dart and uses widgets and hot reload for efficient and fast development. Dart is an object-oriented language with C-style syntax, Dart includes advanced features such as garbage collection for memory management, synchronous and asynchronous handlers, null safety, and compile-time data type checking.

We selected Flutter for its low development cost and Foreign Function Interface (FFI), a Dart plugin that allows the application to invoke functions from pre-compiled C libraries. Flutter and Dart are unique, in that, there are no direct alternatives. Rather, there is a collection of languages and frameworks to achieve a similar implementation, such as React Native and JavaScript or Kotlin or Swift. Each language has its own distinct compilation methods. Therefore, Flutter and Dart are a more suitable choice for this project.

\section{Simple Encrypted Arithmetic Library}

To perform computations on encrypted data, this project utilizes Simple Encrypted Arithmetic Library (SEAL) \cite{sealcrypto}, an open-source homomorphic encryption library developed by Microsoft. The library supports limited arithmetic including addition, subtraction, and multiplication to be performed directly on encrypted data without needing to decrypt it. We chose SEAL primarily for its maturity, prevalent adoption in research, and use in Proof of Presence Share \cite{Lagesse2021-PopShare}.

SEAL supports three Fully Homomorphic Encryption (FHE) schemes: Brakerski-Fan-Vercauteren (BFV) \cite{fan2012-bfv}, Brakerski-Gentry-Vaikuntanathan (BGV) \cite{brakerski2012-bgv}, and Cheon-Kim-Kim-Song (CKKS) \cite{Cheon2017-CKKS}. BFV and BGV are integer-based FHE schemes that allow for computations on encrypted integers, but differ in their ciphertext structure and noise management techniques. BFV encodes the message with the most significant bits of the ciphertext and accumulates more noise growth for each multiplication circuit. When too much noise accumulates, the ciphertext cannot be decrypted. For simpler computations, BFV offers better performance. BGV encodes the message in the least significant bits and manages noise through modulus switching. This difference in noise management makes BGV suitable for deeper computations with many sequential operations. 

CKKS is specifically designed for scenarios where approximate arithmetic is acceptable, and this trade-off allows for more efficient computations on encrypted real numbers. This scheme is efficient in performing computations, by representing real numbers as complex numbers. The encoding allows for efficient arithmetic operations, specifically addition, multiplication, and subtraction. When decoding, the result is transformed into a meaningful representation of the true value, with some degree of error. 

\section{Distance Measures}

To accurately measure the similarity between videos while preserving privacy, this project utilizes Kullback-Leibler Divergence (KLD), Bhattacharyya Coefficient (BC), and Cramer’s Distance (CD). Pop-Share [6] established these metrics as an effective indicator for comparing video similarity. Unlike other probability distribution algorithms that involve complex calculations, these three metrics rely primarily on element-wise addition, subtraction, and multiplication of vectors. This simplicity is crucial for their integration with Fully Homomorphic Encryption, as it allows for efficient computation on encrypted data without decryption.

Kullback-Leibler Divergence (KLD) [7] is a state-of-the-art measure of how one probability distribution diverges from a second, expected probability distribution. A limitation of KLD is that it is not symmetric, in that the divergence from distribution P to Q is not necessarily the same as from Q to P, which may complicate its interpretation.

\begin{equation}
    D_{KL}(P || Q) = \sum_{i} P(i) \log \left( \frac{P(i)}{Q(i)} \right)
    \label{eq:kld}
\end{equation}

Bhattacharyya Coefficient (BC) [8] is a measure that quantifies the amount of overlap between two probability distributions, making it useful in various fields such as pattern recognition and image processing. The coefficient ranges from 0 to 1, where a value closer to 1 indicates higher similarity between the distributions.

\begin{equation}
    BC(P, Q) = \sum_{x} \sqrt{P(x) Q(x)}
    \label{eq:bc}
\end{equation}

Cramer’s Distance (CD) [9] is a measure used to quantify the dissimilarity between two probability distributions. It is based on the concept of the characteristic function and finds applications in statistical inference and hypothesis testing. One limitation of Cramer’s Distance is its sensitivity to sample size, which can result in inaccuracies, particularly when the sample size is small.

\begin{equation}
    D_C(P, Q) = \sqrt{\sum_{i} (P_{i} - Q_{i})^2}
    \label{eq:cd}
\end{equation}

% ========== Chapter 3
 
\chapter{Related Work}

This chapter reviews significant advances in Fully Homomorphic Encryption (FHE) and related methodologies relevant to this project. Section \ref{sec:Related FHE Libraries} begins with an overview of existing FHE libraries. In Section \ref{sec:Related Pop-Share}, we examine the Proof of Presence Share methodology, an innovative application of FHE to compute video similarity. In Section \ref{sec:Related Video Similarity}, we discuss alternative video similarity measures.

\section{Fully Homomorphic Encryption Libraries}

Open-Source Fully Homomorphic Encryption Library, OpenFHE [10], is a state-of-the-art C++ library born from a merger of PALISADE, HElib and HEAAN. OpenFHE supports modern schemes and can be configured to use hardware acceleration.

From existing systems, OpenFHE adapted the modular design of PALISADE [3] with the built-in support of BGV, BFV, CKKS and FHEW schemes. From HElib [2] an efficient homomorphic encryption library primarily focused on BGV and CKKS schemes, pioneered research and early development in this domain. HEAAN [11] was integrated to support arithmetic operations on approximate numbers while other schemes only support integer based arithmetic.

Compared to SEAL, OpenFHE library has an actively growing community with modular integrations of existing systems. The documentation and development guides for OpenFHE are much more modern and comprehensive. However, SEAL is much more established within the research community. Regarding performance, there have not been any peer reviewed publications directly comparing the performance of Microsoft SEAL and OpenFHE.

\section{Proof of Presence Share}
\label{sec:Related Pop-Share}
Proof of Presence Share (Pop-Share) \cite{Lagesse2021-PopShare} introduces a novel approach for accurately detecting similar video scenes with strong privacy guarantees, presenting two applications of fully homomorphic encryption deployed within a client-server architecture and a distributed peer-to-peer network. To accurately detect similar video scenes, Pop-Share proposed a new application and evaluation of the Similarity of Simultaneous Observation (SSO) \cite{Wu2019-SSO}. 

Developed by Wu and Lagesse, SSO was intended to detect streaming Wi-Fi cameras by comparing the probability distribution between network traffic and recording videos on the network. In addition, SSO contributed a computationally efficient way to transform video frames into probability distribution form to be compared by various statistical measures including Jensen-Shannon Divergence (JSD), Kullback-Leibler Divergence, and Dynamic Time Warping (DTW) \cite{Sakoe1978-dtw}. The pre-processing approach of SSO slices the video into one-second segments, computes the total number of the bytes for all frames in each segment, and normalizes the byte count array so that it summates to one. However, JSD and DTW could not be simplified enough to be used in fully homomorphic operations, so Pop-Share leveraged Cramer Distance and Bhattacharyya Coefficient to handle the encrypted similarity score measures.

Pop-Share was developed in 2020, its implementation depended on SEAL 3.3 to handle fully homomorphic operations. Once both videos were preprocessed, the transformation was irreversible, having strong privacy guarantees. In addition, the arrays were encrypted and computed using the plaintext counterpart, resulting in less noise accumulation than when using only ciphertext. In a distributed peer-to-peer application, the initiator shares an encrypted video for the recipient to apply their plaintext array onto the untrustworthy ciphertext. The modified ciphertext was returned to the originator to be decrypted and the summation of the decrypted array represents the corresponding similarity score. This was performed for each similarity score measure. A notable limitation of this approach is that their implementation was tightly coupled with a static version of SEAL, making the results difficult to reproduce.

This method demonstrated that SSO generates a unique, one-way signature of videos that can be used to compare for similarity without access to the video's content. For this project, the results from Pop-Share serve as a baseline comparator for the modular Flutter re-implementation of the Pop-Share peer-to-peer application.

\section{Video Similarity}

As a part of SSO and Pop-Share, a video was represented in probability distribution form, in order to compare two probability distributions, there are many methods to produce a similarity score. In this section, we cover all considerations for generating a similarity score, as well as, acknowledge alternative approaches to comparing videos.

Cramer’s Distance (CD) [15] is a measure used to quantify the dissimilarity between two probability distributions. It is based on the concept of the characteristic function and finds applications in statistical inference and hypothesis testing. One limitation of Cramer’s Distance is its sensitivity to sample size, which can result in inaccuracies, particularly when the sample size is small.
Bhattacharyya Coefficient (BC) [16] is a measure that quantifies the amount of overlap between two probability distributions, making it useful in various fields such as pattern recognition and image processing. The coefficient ranges from 0 to 1, where a value closer to 1 indicates higher similarity between the distributions.

Kullback-Leibler Divergence (KLD) [17] is a widely used measure of how one probability distribution diverges from a second, expected probability distribution. A limitation of KLD is that it is not symmetric, meaning that the divergence from distribution P to Q is not necessarily the same as from Q to P, which may complicate its interpretation.
Jensen-Shannon Divergence (JSD) [18] is a symmetric version of KLD that quantifies the similarity between two probability distributions. It is widely used in information retrieval and clustering. A limitation is that while it provides a measure of similarity, it may not capture finer details of the distributions being compared.

The Pearson Correlation Coefficient (PCC) [19] measures the linear correlation between two variables, producing a value between -1 and 1. It is a fundamental statistical tool in various fields, including psychology and economics. However, this coefficient assumes a linear relationship and may not capture non-linear associations, which could lead to misleading conclusions in certain contexts. 
Dynamic Time Warping (DTW) [20] is an algorithm used to measure similarity between two temporal sequences that may vary in speed. It has applications in speech recognition, data mining, and video analysis. One limitation of DTW is its computational complexity, particularly for long sequences, which can make it less suitable for real-time applications. 
Alternatives to SSO are advanced applications on top of existing similarity score algorithms. These novel approaches are typically not restricted to resource-constrained environments.

Using content-based algorithms, Shan and Lee [21] presented a series of optimal mapping algorithms to detect similarity between video frames of unequal sizes. In addition, Wu, Zhuang, and Pan [22] present a similar shot graphing algorithm with the intention of identifying similar videos within a large database of videos. An enhancement of Bhattacharyya, was proposed by Loza, Mihaylova, Canagarajah, and Bull [23] to generate a particle filter that would iterate over each sample to predict and resample based on the similarity. Overall, these methods may be adapted to be implemented with fully homomorphic encryption, however they would need to be simplified to use basic arithmetic, and may require many parameters to produce a score.


% ========== Chapter 4
 
\chapter{Design}

The core design principle is modularity, with each component designed to support additional cryptographic libraries and be flexible for future applications. This project is comprised of three components: two Dart plugins and a Flutter application. In Section \ref{sec:Fully Homomorphic Encryption Library}, the Fully Homomorphic Encryption (FHE) Library plugin integrates Microsoft SEAL and delivers FHE arithmetic for the Dart and Flutter communities. In Section \ref{sec:Distance Measure Plugin}, the Distance Measure plugin implements Kullback-Leibler Divergence, Cramer Distance, and the Bhattacharyya Coefficient, along with simplified versions of their original algorithms developed for basic FHE arithmetic operations. In Section \ref{sec:Video Similarity Application}, GhostPeerShare depends on both Dart plugins as a peer-to-peer Flutter application.

\section{Fully Homomorphic Encryption Library}
\label{sec:Fully Homomorphic Encryption Library}
This implementation adheres to the core design principle of modularity, remaining agnostic to the underlying C++ library. Instead of directly invoking methods within the plugin, the plugin uses the Bridge and Adapter structural design patterns inspired by the work of Alberto Ibarrondo and Alexander Viand \cite{Ibarrondo2021-Pyfhel}. The Bridge pattern decouples the high-level Dart API from the specific C++ implementation by defining an abstract interface. The Adapter pattern connects the Dart interface to the C++ implementation through a C-based intermediary, translating Dart requests into operations understood by the native library. This approach enables compatibility with multiple Fully Homomorphic Encryption (FHE) backends without modifying the Dart API.

The Fully Homomorphic Encryption Library provides a modular API that integrates Microsoft SEAL with the Flutter ecosystem. It offers a straightforward interface for developers, abstracting low-level complexities while delivering access to advanced encryption functionalities required for secure applications. The topology of this Dart plugin ensures a clear separation of responsibilities to promote maintainability and extensibility. Table \ref{table:class-structure} outlines the distinct roles of each component, including the high-level API, the adapter interface, and the concrete implementation.

\begin{table}[t]
\caption{Class Structure Overview}
\centering
\begin{tabular}{|l|l|p{9cm}|}
\hline
\textbf{File} & \textbf{Class} & \textbf{Description} \\ \hline
seal.dart    & Seal    & Entry-point for the end-user API in Dart \\ \hline
afhe.dart    & Afhe    & Dart adapter connecting to the C interface \\ \hline
fhe.cpp      & N/A     & C interface bridging Dart and C++ \\ \hline
afhe.h       & Afhe    & Pure abstract class defining the FHE contract \\ \hline
aseal.cpp    & Aseal   & Concrete implementation of the Afhe abstraction \\ \hline
\end{tabular}
\label{table:class-structure}
\end{table}


Each file plays a specific role in this architecture. The \textit{seal.dart} file provides the primary entry point for the end-user API, allowing developers to interact with the encryption library in Dart. The \textit{afhe.dart} file acts as the adapter, invoking lower-level C functions. The Afhe class exposes abstracted data types such as Keys, Ciphertext, and Plaintext objects. The \textit{fhe.cpp} file defines the C interface. It exposes the inherited methods from \textit{afhe.h} and references the memory address of the underlying C++ object. Depending on the reference, the corresponding concrete implementation will be invoked. For example, \textit{aseal.cpp} manages Microsoft SEAL objects.

The Foreign Function Interface (FFI) is this plugin's core dependency; It enables Dart to interact with the pre-compiled C binary. For each target platform, the dynamically linked library must be accessible on the local file system. This plugin exposes methods to cast native C data types into Dart objects and vice versa. Our implementation creates Dart objects that mirror their corresponding underlying C++ class. For example, in Microsoft SEAL, a Ciphertext can be saved or loaded. We expose \textit{save\_ciphertext} and \textit{load\_ciphertext} in the C interface. In Dart, we create a Ciphertext class that contains both \textit{save} and \textit{load}, referencing the memory address of the underlying abstract Ciphertext object. This implementation pattern is consistent and transparent for security researchers familiar with the underlying C++ API. For new developers, our implementation mirrors the structure of Microsoft SEAL, providing a clear and familiar interface.

In order to prevent breaking changes, unit tests ensure that the functional behavior of the library remains consistent and reliable throughout the development and release. Unit tests serve as a safety net. They ensure that existing features function as expected when new changes occur. This process reduces the risks of regressions. For SEAL, the authors developed a set of examples that walk the user through their APIs. As a part of the unit tests, we re-implemented these examples with our interface to increase our confidence that our implementation did not introduce any new or unexpected behaviors. For automation, GitHub Actions compiles and executes unit tests using GoogleTest \cite{GoogleTest} with CMake. In total, There are 78 unit tests: 35 in C and 43 in Dart.

In order to distribute this plugin to Linux and multiple Android flavors, GitHub Actions employs Conan \cite{Conan}. Conan automates the management of dependencies and packaging, specifically addressing the versioning of Microsoft SEAL and other backend libraries. For Linux, Conan streamlines the building and packaging process by automatically resolving dependencies tailored to various distributions (e.g., Ubuntu, Fedora, and CentOS) while ensuring compatibility with different versions of Microsoft SEAL and system architectures (e.g., x86, ARM). Similarly, Conan simplifies cross-compilation for Android by allowing developers to specify configurations for various architectures and API levels, ensuring that the plugin can be compiled for the supported CPU architectures x86\_64, ARMv8, and ARMv7. The compiled binaries are all hosted on all platforms and are available for download for each release of the Dart plugin. When developers add the plugin to their dependencies, the binaries are pulled down from GitHub and stored within their local, versioned dependency cache. This automated approach enhances the consistency and reliability of the plugin across different environments, facilitating smoother deployment and easier updates for users.
\section{Distance Measure Plugin}

This Dart plugin calculates three distance measures: Kullback-Leibler Divergence, Cramer Distance, and Bhattacharyya Coefficient, as well as their Fully Homomorphic Encryption (FHE) counterpart. To manage the complexity of these computations, the plugin only supports operations on lists of plaintext doubles, which are then encrypted for processing. In this section, we explain the fully homomorphic computations for each algorithm, along with details on testing and distribution.

%TODO%
Fully homomorphic encryption (FHE) enables computations on encrypted data, ensuring that data privacy is maintained throughout the process. Each algorithm’s encrypted score remains interpretable because it retains the mathematical relationships needed for distance calculation without exposing the underlying plaintext values. For more information on the equations of the plaintext scores, refer to Section 2.3. However, a major limitation in FHE schemes is their inability to directly support division due to the mathematical structures in FHE, which operate within polynomial or modular arithmetic. Workarounds, such as using multiplicative inverses, do exist, though they add complexity and are unreliable.

To accommodate the limitations of FHE, we cannot directly encrypt the input array of doubles for score computation. Instead, we adjust certain parameters tailored to each algorithm’s requirements before encryption. The encrypted parameters and any associated data are stored as binary files, with each algorithm (e.g., $kld\_logX$) using 60 binary files, each representing one second of a one-minute video. This structure is particularly helpful for cases where video lengths differ, as it simplifies the trimming process. The binary files are then archived into an .enc file format, which is unique to our application and contains both the encrypted data and a metadata JSON file with relevant video details (e.g., timestamp, length, frames per second). The archive averages around 2 MB on Linux for 1 minute of video.

To compute the encrypted Kullback-Leibler Divergence using FHE, noted as $KLD_{enc}$ in Equation \ref{eq:fhe_kld_enc}, the ciphertext modification requires three parameters: $X$, $logX$, and $Y$, where $X$ and $logX$ are encrypted arrays, and $Y$ is plaintext. The modified ciphertext is obtained by multiplying each encrypted double in $X$ by the difference between $logX$ and $logY$. This yields an encrypted list of doubles, which, when decrypted and summed, results in the divergence of $Y$ from $X$, shown in Equation \ref{eq:fhe_kld_decrypt}.

\begin{equation}
        KLD_{enc} = ( \log(X)_{enc}^i - \log(Y_{plain}^i) ) \cdot X_{enc}^i
    \label{eq:fhe_kld_enc}
\end{equation}

\begin{equation}
        KLD_{plain} = \sum_{i=1}^{n} D_K(KLD_{enc})
    \label{eq:fhe_kld_decrypt}
\end{equation}

For the Bhattacharyya Coefficient, indicated as $BC_{enc}$ in Equation \ref{eq:fhe_bc_enc}, the square root of $X$ is taken before encryption. This calculation involves two arrays, $sqrtX$ and $Y$, both floating-point arrays. We multiply each encrypted double in $sqrtX$ by the square root of $Y$, yielding an encrypted list. Decrypting and summing this list gives the measure of overlap between the two probability distributions, shown in Equation \ref{eq:fhe_bc_decrypt}.

\begin{equation}
    BC_{enc} = (\sqrt{X})_{enc}^i \cdot \sqrt{Y_{plain}^i}
    \label{eq:fhe_bc_enc}
\end{equation}

\begin{equation}
    BC_{plain} = \sum_{i=1}^{n} D_K(CD_{enc})
    \label{eq:fhe_bc_decrypt}
    \title{Foo}
\end{equation}

To calculate the Cramer Distance, or $CD_enc$ in Equation \ref{eq:fhe_cd_enc}, the cumulative distribution of $X$ is computed iteratively so that the last element sums to one. This requires two arrays, $X$ and $Y$, of floating-point numbers. The computation squares the difference between corresponding elements in $X$ and $Y$, resulting in an encrypted list of values, which, when decrypted, summed, and square-rooted, provides the distance between the distributions, shown in Equation \ref{eq:fhe_cd_decrypt}.

\begin{equation}
    CD_{enc} = (X_{i} - Y_{i})^2
    \label{eq:fhe_cd_enc}
\end{equation}

\begin{equation}
    CD_{plain} = \sqrt{\sum_{i=1}^{n} D_K(CD_{enc})}
    \label{eq:fhe_cd_decrypt}
\end{equation}

The plugin undergoes automated testing via GitHub Actions, which validates scores generated from identical input data to within a 7-decimal-point precision (as shown in Table \ref{table:mean-error}, which details the noise introduced by FHE). Testing covers both the symmetric and asymmetric behavior of each algorithm. A core design goal of this library is modularity, allowing for the future integration of additional distance measures using FHE. This work introduces the FHEL Dart plugin, creating a foundational Dart plugin for distance measurements that supports fully homomorphic encryption.

\section{Video Similarity Application}
To perform computations on encrypted floating-point numbers, we utilize the CKKS scheme [11] with a polynomial modulus degree of 4096. CKKS is a homomorphic encryption scheme to perform computations of approximate arithmetic on real numbers. This scheme offers out-of-the-box support for decimal value computations compared to other schemes like BFV and BGV, which primarily focus on integer arithmetic. The choice of 4096 as the polynomial modulus degree balances the need for strong security and accuracy with practical computational performance. While a larger degree enhances security and allows for more operations before result degradation, with the trade off of increasing computation time and ciphertext size.

In order to detect similarities between two videos, we represent each video as a probability distribution of byte size changes between one-second segments, shown in Figure \ref{fig:preprocess-data-flow}. To transform raw video into comparable probability distributions, we count the number of bytes in each frame, calculate the sum of frame lengths in each segment, and then normalize the array of segments between zero and one, known as a Probability Distribution Form (PDF), ensuring that the distributions for both videos have the same scale and can be directly compared. This approach is borrowed from the Similarity of Simultaneous Observation [12], and provides strong privacy guarantees, as if the encryption is broken, the raw byte data cannot be reconstructed from the normalized byte array due to the loss of information during the aggregation and normalization process. We have considered a motion estimation approach with the Lucas-Kanade method [23] supported by OpenCV, which could provide more detailed information about video content. However, for this study, it is crucial to compare our approach to the established baseline metrics, and new methods would hinder this comparison.

\begin{figure}[ht]
    \centering
    \includegraphics[width=\textwidth]{4 Design/4.3 Preprocess Data Flow.png}
    \caption{Pre-process Data Flow}
    \label{fig:preprocess-data-flow}
\end{figure}

To compute multiple similarity scores using homomorphic encryption, separate byte arrays are encrypted for each score. The recipient can perform computations with plaintext decimal values against the corresponding encrypted element in the array. This approach leverages the unique ability of fully homomorphic encryption and was borrowed from Pop-Share, as this approach is much faster than comparing ciphertext against ciphertext. 
For Kullback-Leibler Divergence, we share the PDF and the natural log of the PDF. For each encrypted double in the PDF, we subtract the natural log of the plaintext element and multiply the difference. The product remains encrypted and returned to the originator, where it can be decrypted and the summation is the similarity score.
For Bhattacharyya Coefficent, we share the square root of the PDF. For each encrypted double, we multiply the ciphertext by the square root of the plaintext element. The product remains encrypted and returned to the originator, where it can be decrypted and the summation is the similarity score.
For Cramer Distance, we share Cumulative Distribution Form (CDF) from PDF. The CDF is the cumulative sum of all of the elements in the PDF, such that the last value of CDF is equal to 1. For each encrypted double, we square the difference of the plaintext element. The product remains encrypted and returned to the originator, where it can be decrypted and the square root of the summation is the similarity score.


% ========== Chapter 5

\chapter{Results}

In this chapter, we compare our implementation to the foundational contributions of SSO and Pop-Share. In order to evaluate the accuracy of the Fully Homomorphic Encryption Library plugin in Dart, we measure the amount of noise generated during encrypted operations in Section \ref{sec:Noise Accumulation}. In Section \ref{sec:SSO Distance Measure}, we compare our implementation of our distance measures using the original pre-processed video frames from SSO. In Section \ref{sec:Field of View}, we compare the same video from various angles and determine the accuracy of our implementation. In Section \ref{sec:Scene Distance Measure}, we compare the distance measures for similar scenes against different scenes. In Section \ref{sec:Performance}, we compare the performance across multiple Linux and Android devices.

\section{Noise Accumulation}

When performing fully homomorphic operations, a small amount of noise is introduced that affects the accuracy of the decrypted distance measure. If too much noise is accumulated, the plaintext cannot be recovered. Table \ref{table:mean-error} represents the baseline Mean Error from Pop-Share \cite{Lagesse2021-PopShare}, compared to our implementation. For Kullback-Leibler Divergence, Cramer Distance, and Bhattacharyya Coefficient our implementation introduced half the amount of noise than the original.

As illustrated in Figure \ref{fig:pop-share-distance-measure}, our Dart implementation exhibits less variance compared to the Python baseline, allowing for a clearer distinction in mean error scores. This reduced variance indicates a more stable performance in our implementation, reinforcing the reliability of the distance measures employed. Furthermore, through automated unit tests, we have ensured that the scores do not differ more than $1.0e-7$ epsilon, confirming the consistency and accuracy of our calculations.

\begin{table}[t]
\caption{Difference of Mean Error for each algorithm}
\centering
\begin{tabular}{|c|c|c|c|}
\hline
\textbf{Function} & \textbf{Pop-Share ($PS$)} & \textbf{GhostPeerShare ($GPS$)} & \textbf{Difference ($PS-GPS$)} \\
\hline
KLD     & $4 \times 10^{-9}$ & $1.71 \times 10^{-11}$ & $3.98 \times 10^{-9}$ \\
Cramer  & $4 \times 10^{-4}$ & $1.61 \times 10^{-9}$ & $3.99 \times 10^{-4}$ \\
BC      & $8 \times 10^{-4}$ & $3.85 \times 10^{-10}$ & $7.99 \times 10^{-4}$ \\
\hline
\end{tabular}
\label{table:mean-error}
\end{table}


Finally, the parameter choices within the encryption scheme-such as polynomial modulus degree, encoder scalar, or array of modulus sizes-also play a significant role in noise management.


\input{5 Results/5.2 SSO Distance Measure}
\section{Field of View}
Borrowing the methodology from Pop-Share, we compare videos without revealing their content, manually labeling each row of similarity scores with a one or zero to train a supervised artificial neural network for classifying new scenes. The binary labeling system assigns a one when two scenes are similar or identical; otherwise, a zero is assigned. In Dart, we exhaustively compare the plaintext byte count arrays to generate a row of labels along with the Kullback-Leibler Divergence (KLD), Bhattacharyya Coefficient (BC), and Cramer Distance (CD) for comparison between each one-second segment of videos of the same length. Prior to comparison, both videos have been manually aligned and trimmed to ensure consistency.

\begin{table}[b!]
\centering
\begin{tabular}{lccccc}
\hline
\textbf{System} & \textbf{F1 (\%)} & \textbf{Precision (\%)} & \textbf{Recall (\%)} & \textbf{Accuracy (\%)} & \textbf{Error (\%)} \\
\hline
PoP-Share[2]   & 96.63 & 97.73 & 95.56 & 95.16 & 4.84 \\
SSO[6]         & 96.13 & 92.56 & 100.00 & 96.30 & 3.70 \\
Handheld       & 97.97 & 99.32 & 96.67 & 98.00 & 2.00 \\
GhostPeerShare & 97.09 & 96.14 & 98.05 & 98.05 & 1.95 \\
\hline
\end{tabular}
\caption{Performance comparison of different systems.}
\label{table:ann_system_compare}
\end{table}



As shown in Table \ref{table:ann_system_compare}, we contribute GhostPeerShare to the previous analysis conducted in Pop-Share by employing supervised learning with an artificial neural network to accurately classify a binary score, representing the same or different scene based on the three distance measures. We trained and tested the model on a small dataset of 100 minutes of raw videos, which included three twenty-minute videos of low movement featuring an individual sitting at a desk, as well as two twenty-minute videos of high movement capturing an individual vacuuming his living room. The best results from all training and testing scenarios, detailed in Appendix Table \ref{table:appendix_ann}, achieved an accuracy and recall of 99.95\%, a precision of 100\%, and an F1 Score of 99.98\%. This performance was attained using a grid search, with training data based on the raw videos and testing conducted solely on comparisons of different scenes, where the label is exclusively zero.

\section{Scene Distance Measure}
\label{sec:Scene Distance Measure}

The purpose of this experiment is to investigate the range of distance measure scores from our training data in Section \ref{sec:Field of View}. Figure \ref{fig:scene-distance-measure} illustrates the average distance measure score for each algorithm from our dataset, comparing different and same scenes. The scenes were labeled according to our binary labeling system, where one indicates similar scenes and zero otherwise. The graph also includes the standard error above and below the mean distance measure, indicating the range within which the true mean distance measure score is expected. This graph mirrors the same experiment performed by Pop-Share. Pop-Share's training data was much more diverse, with around 16 hours of raw video spanning across various environments and recorded from many devices. GhostPeerShare was training on a total of 100 minutes of indoor video data recorded by two phones.

\begin{figure}[t]
    \centering
    \includegraphics[width=0.8\textwidth]{5 Results/Figures/5.4 Bar Chart.png}
    \caption{Scene Distance Measure}
    \label{fig:scene-distance-measure}
\end{figure}

For the Bhattacharyya Coefficient, a perfect score of 1.00 indicates that the two scenes are identical, while a score of zero denotes no overlap. From Pop-Share, the same pattern was found for the Bhattacharyya Coefficient, containing little variance between different and same scenes and significantly less standard error than other metrics. In our experiments, we observed a 0.02 variance between the distance measures for different and same scenes. This minimal variance is likely attributed to the uniformity of the scene types recorded, as both sets were filmed indoors, and the fact that the same devices were used during the experiments, which limited the diversity in the captured footage.

For Cramer Distance, a perfect score of 0.00 indicates that the two scenes are identical, while a score of 1.0 is the maximum distance between two probability distributions. From Pop-Share, Cramer Distance contained a significant absolute difference of around 0.25 between different and same scenes. In our experiments, we observed significantly less variance, 0.02, between the distance measures for different and same scenes. The lack of variance is likely attributed to the limited data used in our experiments compared to Pop-Share. 

For the Kullback-Leibler Divergence (KLD), a perfect score of 0.00 indicates that two scenes are identical, while the upper limit is infinity; the smaller the score, the more similar the two probability distributions are. From Pop-Share, KLD was the leading indicator of dissimilarity, however, their experiments contained significant variance greater than 1 between different and same scenes. In our experiments, we observed a 0.34 variance between the distance measures for different and same scenes. This represents the highest variance observed among all of our distance measures and serves as a significant indicator that two scenes may be different.

The key takeaway from our experiments is that while the individual scores for the distance measures exhibit minimal variance, they still effectively differentiate between similar and different scenes. This consistency suggests that the implemented algorithms can reliably assess similarity, even in scenarios where the overall variation in scores is low. Consequently, the results highlight the robustness of our methodology in accurately identifying scene differences, which is crucial for applications requiring precise content analysis without revealing the underlying video data.

\section{Performance}
\label{sec:Performance}
To compare this implementation to the mobile Proof of Presence Share (Pop-Share) application, the preprocessing speed for one-minute videos was benchmarked, and the duration of encrypted operations was measured against plaintext operations. This comparison allowed for assessing the implementation's efficiency and quantifying the impact of encryption on performance.

\begin{table}[t]
\caption{Pre-processing duration (in seconds) for a one-minute video.}
\centering
\begin{tabular}{|l|c|c|c|}
\hline
\textbf{System} & \textbf{Average (s)} & \textbf{Min (s)} & \textbf{Max (s)} \\
\hline
PC              & 111.25 & 13.00 & 321.00 \\
Samsung S9      & 421.62 & 180.00 & 755.50 \\
Pixel 3XL       & 454.50 & 171.00 & 875.00 \\
\hline
\end{tabular}
\label{table:preprocessing_times}
\end{table}
 

Experiments used one-minute videos with h264 encoding, results shown in Table \ref{table:preprocessing_times}. On a Pixel 3 XL mobile phone, preprocessing with OpenCV required 438 seconds (about 7.3 minutes) on average. In contrast, Pop-Share on a Pixel 2 took only 17 seconds on average using FFMpeg. This difference stems from the use of different libraries for preprocessing. Limited operating system support for FFMpeg plugins restricted plugin selection to one compatible with both Linux and Android. On a Linux PC with OpenCV, preprocessing took 111 seconds (about 2 minutes) on average. This result demonstrates the PC operated approximately four times faster than the average of the two mobile devices. The disparity likely arises from differences in CPU thread availability and speed. The PC has 32 threads at 3.4 GHz, while both Android phones have 8 threads at about 2.5 GHz.

Dart code executes in isolates, which resemble threads but with separate memory allocations. Flutter applications perform all operations on a single isolate. Although the implementation creates an isolate to count bytes in each one-second segment, it does not incorporate parallelism. The frame parsing process remains sequential. This design choice prioritized stability over performance to ensure consistent operation. Table \ref{table:preprocessing_times} presents the execution times on mobile and Linux desktop environments. The data reveals substantial differences that highlight inefficiencies in our mobile implementation.

The encryption and encoding of each distance measure varies, as the parameters to the Fully Homomorphic distance measure implementation are different, described in further detail in Section \ref{sec:Distance Measure Plugin}. Unfortunately, we cannot compare these metrics to Pop-Share as they were not explicitly mentioned in their evaluation. However, our approach treats the preprocessing of video frames as an independent transaction, separate from the encoding and encryption steps. Shown in Table \ref{table:cryptography}, the Kullback-Leibler Divergence (KLD) encryption is about twice that of Bhattacharyya and Cramer due to the requirement of double the amount of data. This pattern was also indicated in Pop-Share.

\begin{table}[h!]
\centering
\begin{tabular}{|l|c|c|c|}
\hline
\textbf{Algorithm} & \textbf{Encryption (ms)} & \textbf{Decryption (ms)} \\
\hline
KLD & 1019.00 & 229.88 \\
BC  & 503.00 & 140.25 \\
CD  & 507.12 & 217.50 \\
\hline
\end{tabular}
\caption{Comparison of Cryptography (in Milliseconds) on Android}
\label{table:cryptography}
\end{table}


The Fully Homomorphic Encryption computations, shown in Table \ref{table:fhe_operations}, are within ±20\% of Pop-Share, validating our abstraction design and interface implementation. This performance increase is likely due to the advancements in Microsoft SEAL. Pop-Share used version v3.3, while our implementation uses v4.1.

\begin{itemize}
    \item Kullback-Leibler Divergence: Pop-Share reported an average time of 400 milliseconds, whereas our implementation achieved 346 milliseconds, representing a 13.5\% increase in performance.
    \item Cramer Distance: Pop-Share reported an average time of 386 milliseconds, whereas our implementation achieved 310 milliseconds, representing a 19.4\% increase in performance.
    \item Bhattacharyya Coefficient: Pop-Share reported an average time of 186 milliseconds, whereas our implementation yielded 203 milliseconds, representing a 9.1\% decrease in performance.
\end{itemize}

\begin{table}[h!]
\centering
\begin{tabular}{|l|c|c|c|}
\hline
\textbf{Algorithm} & \textbf{FHE Compute (ms)} & \textbf{Plaintext Compute (ms)} \\
\hline
KLD & 345.38 & 1.12 \\
BC  & 202.38 & 0.41 \\
CD  & 309.38 & 0.45 \\
\hline
\end{tabular}
\caption{Comparison of FHE Operations (in Milliseconds) on Android}
\label{table:fhe_operations}
\end{table}


The plaintext computations, shown in Table \ref{table:fhe_operations}, are much faster than expected, a significant performance increase in Dart compared to Python implementation. This performance increase may be due to the underlying data structure selection in Dart.

\begin{itemize}
    \item Kullback-Leibler Divergence: Pop-Share reported an average time of 3.8 milliseconds, whereas our implementation achieved 1.12 milliseconds, representing a 70.5\% increase in performance.
    \item Cramer Distance: Pop-Share reported an average time of 5.2 milliseconds, whereas our implementation achieved 0.45 milliseconds, representing a 91.35\% increase in performance.
    \item Bhattacharyya Coefficient: Pop-Share reported an average time of 3.2 milliseconds, whereas our implementation achieved 0.41 milliseconds, representing an 87.5\% increase in performance.
\end{itemize}

To extract a distance measure from the archive, we calculate the score by decrypting each ciphertext in the modified byte count array and summing all the elements. Cramer's method differs as it involves calculating the square root of the result. Shown in Table \ref{table:cryptography}, Pop-Share had an average duration of 250 milliseconds, while our Android implementation took 588 milliseconds. Although the difference is insignificant, it suggests potential differences in our implementation.

In total, on Android, uploading and converting the video frames into byte count arrays took, on average, 438 seconds (7 minutes) for a one-minute video. Encoding, encrypting, and generating the archive took, on average, 2 seconds. When performing Fully Homomorphic Encryption (FHE) operations on all ciphertexts took, on average, 857 milliseconds. Finally, decryption took, on average, 588 milliseconds. In total, it took 3.4 seconds to derive a similarity score using FHE. As the preprocessing is expensive, it only has to be performed once for every video uploaded to the application.

Overall, this study benchmarked the performance of our implementation on Android to the existing Pop-Share application. The results showed significantly slower preprocessing times due to library limitations, but the encryption, computation, and decryption times were competitive with Pop-Share. Our implementation demonstrates the high efficiency of the Fully Homomorphic Encryption operations. Notably, plaintext computations were significantly faster in our implementation.


% ========== Chapter 6

\chapter{Discussion}
\section{Qualitative Analysis}

%   - Programmer usability - how would you measure that?
% 	  - Developer Surveys?
%   	* Learnability Metrics (Time to complete example, number of docs/references used)
%   	* Likert-scale questions (rate the clarity of API / doc from 1 to 5)
% 	  - Adoption Rate (pub.dev, Github)

In order to address the usability of our Flutter application and Dart plugins, this section aims to propose methods of gathering qualitative metrics of our implementation that benefit the Flutter community and security researchers. There are two audiences we wish to target in this analysis: 1) Developers and 2) Security Researchers. Developers may be enthusiasts or professionals seeking access to integrate Fully Homomorphic Encryption within an existing or new application. Security Researchers may be trying to apply Fully Homomorphic Encryption to a novel problem or integrate a new C++ library into our FHEL plugin. To reach our target audience, we may employ developer surveys to gather data on consumers of our plugins, monitor the adoption rate using open-source tools, and publish a workshop paper to advertise and peer-review our implementation utilizing reputable journals.

For developer surveys, we aim to gather information about each individual's programming knowledge or familiarity. Using a Likert-scale questionnaire, we may rate the clarity of our API documentation on a scale of 1 to 5. Using open-ended questions to capture insightful feedback from users who may be seeking additional functionality or identify gaps in our implementation. This structured survey format enables maintainers to assess the level of difficulty in integrating the plugins relative to each individual's experience with open-source projects. Gathering data on the consumers of our packages will help us understand their use cases and the greater impact of this project within the Flutter community. These surveys may be available on the distribution platforms the packages are hosted on, specifically GitHub and Pub.Dev.

To capture the adoption rate of our packages, we may leverage the platforms on which they are hosted. The source code for this project is available on GitHub. Using the star functionality is a popular method of endorsing a project to receive notifications on development updates. The maintainers of this project may visualize the number of stars over time to measure the project's popularity.

Pub.Dev is the official package repository for the Dart and Flutter community. The repository contains a comprehensive set of scores to measure the quality of each package, specifically popularity and pub points. To approximate the popularity of a package within the last 60 days, the repository contains a thumbs-up mechanism similar to the GitHub star button. This button is used as a quantitative metric of the number of developers who like the package. In addition, the number of downloads from distinct callers and the number of dependents contribute to the popularity score. A unique qualitative score, known as Pub Points, evaluates packages across five categories: Dart file conventions, documentation quality, platform support, static analysis, and support for up-to-date dependencies. The maximum achievable score is 160 points.

The $fhel$ plugin received a Pub Score of 140 pub points, 1 likes, and a 27\% popularity score. On GitHub, the $fhel$ repository has received 5 stars from the community. The $fhe\_similarity\_score$ plugin has received a Pub Score of 150 points, 1 likes, and 42\% popularity score. On Github, the repository has received 1 star from the community.

\section{Security Analysis}

%   - Summarize from Pop-Share
%   - Extend or restate...

GhostPeerShare implements the methodology from Pop-Share as a Flutter application. The authors of Pop-Share presented two key insights in their security analysis: 1) Symmetric Key Cryptographic Equivalency and 2) Brute Force.

First, the security of Fully Homomorphic Encryption (FHE) is equivalent to that of other symmetric key cryptosystems, and its strength depends on the chosen parameters. The security equivalency of FHE is a function of the ring dimension and the size of the ciphertext modulus \cite{Albrecht2021-standard}. By default, GhostPeerShare and Pop-Share balance security and performance by selecting a polynomial modulus of 4096, which is equivalent to a symmetric-key cryptosystem with a 128-bit key, such as the Advanced Encryption Standard (AES-128).

Second, a brute-force attack is unlikely due to the high precision score of the SSO-based system, as demonstrated in Section \ref{sec:Field of View}. An attacker attempting to input a random video is highly unlikely to produce erroneous results. Since the attacker cannot ascertain how close their video was to being accepted as co-present, the information they can derive regarding the proximity of their video to acceptance is minimal. From a practical perspective, users may choose to discontinue interaction after a certain number of failed attempts.

GhostPeerShare uses archives to package all ciphertext objects and metadata for data sharing between two nodes. However, these archives do not include signature validation and may be vulnerable to tampering. To mitigate this risk, a signature can be derived from the contents of the archive and sent as part of the payload. The recipient can then validate whether the signature matches their computation. From a practical standpoint, this solution requires users to decide how they share the archive with the recipient.

On Android, QuickShare \cite{samsung_quick_2020} employs end-to-end encryption to secure data in transit. Samsung has also released Private Share for newer Android devices, allowing users to configure a time-to-live (TTL) for shared files. From a practical perspective, users can revoke access to their shared archive after a designated duration with no response.


% ========== Chapter 7

\chapter{Conclusion}
GhostPeerShare significantly advances the accessibility of Fully Homomorphic Encryption (FHE) within the mobile community. Key contributions demonstrate the efficiency of the Fully Homomorphic Encryption Library (FHEL) and ease of integration into another Dart plugin or within a Flutter application.
To measure the efficiency of our application, we benchmark GhostPeerShare against Proof of Presence Share (Pop-Share).
When preprocessing videos, GhostPeerShare generated the byte-count array 7 minutes slower. However, our implementation may be improved to multi-threading or replaced with the same underlying FFmpeg video processing library as Pop-Share.
When trained on a dataset with continuous movement and tested on a dataset with infrequent movement, the application achieved 0.9614 precision and an F1 score of 0.9709. These results indicate that GhostPeerShare consistently generated a distinct representation of each video in the dataset.
When executing FHE operations, GhostPeerShare demonstrated similar performance to Pop-Share. The computational times of FHE distance measure under CKKS reveal that our implementation was within 10\% of Pop-Share. These results suggest that the FHEL plugin maintains Microsoft SEAL's high performance.
Overall, GhostPeerShare re-implements Pop-Share with efficient FHE computation. However, the application requires significant improvement in pre-processing implementation. 

A significant limitation of FHEL is its versioning strategy. Currently, the plugin supports Microsoft SEAL; however, when extended to support additional libraries, changes to any backend library require a minor version change. Rather than deploying a monolith plugin, the version system should deploy multiple artifacts from the same shared codebase, for example, fhel\_seal and fhel\_openfhe. The proposed version system allows each FHEL plugin to align its version numbers with the underlying libraries' versions, simplifying version management for package maintainers and consumers.

Ultimately, this project lays the foundation for future applications of FHE in mobile environments, especially in fields like healthcare and secure data exchange, where privacy and accuracy are paramount. By making SEAL accessible within a mobile-compatible, modular framework, we are opening pathways for broader adoption of FHE and setting the stage for innovation in mobile security.


%
% ==========   Bibliography
%
\nocite{*}   % include everything in the uwthesis.bib file
\bibliographystyle{ieeetr}
\bibliography{uwthesis}
%
% ==========   Appendices
%
\appendix
\raggedbottom\sloppy
 
% ========== Appendix A
 
\chapter{Supervised Learning}

In this section, we present all of the configurations of the Supervised Artificial Neural Network experiments. We trained and tested the model on a small dataset of 100 minutes of raw videos, which included three twenty-minute videos of low movement featuring an individual sitting at a desk, denoted as \textit{office}, and two twenty-minute videos of high movement capturing an individual vacuuming his living room, denoted as \textit{vaccum} in Appendix Table \ref{table:appendix_ann}. We label \textit{default} representing the default parameters of the Multi-Layer Perceptron (MLP) Classifier \cite{scikit-learn}. The MLP Classifier used the Limited-memory Broyden-Fletcher-Goldfarb-Shanno (lbfgs) solver to apply weights within the neural network. The model was trained with two hidden layers, each with 10 neurons, and used Rectified Linear Unit (relu), a popular activation function in deep learning. Labeled as \textit{gridSearch}, a model fitting algorithm. Due to the visual similarity of the videos filmed indoors from the same devices, the grid search performs very well on training data but poorly on additional test data. Due to this overfitting pattern with grid search on our small dataset, we only considered results where we trained on one scene and tested on a different scene to ensure we have proper diversity.
 
\begin{table}[h!]
\centering
\begin{tabular}{lcccc}
\hline
\textbf{Configuration} & \textbf{Accuracy (\%)} & \textbf{Precision (\%)} & \textbf{Recall (\%)} & \textbf{F1-Score (\%)} \\
\hline
office\_default                   & 97.42 & 94.90 & 97.42 & 96.14 \\
office\_gridSearch                & 99.85 & 99.85 & 99.85 & 99.85 \\
vacuum\_default                   & 97.58 & 95.21 & 97.58 & 96.38 \\
vacuum\_gridSearch                & 99.65 & 99.66 & 99.65 & 99.64 \\
office\_vs\_vacuum\_default       & 99.82 & 100.00 & 99.82 & 99.91 \\
office\_vs\_vacuum\_gridSearch    & 99.95 & 100.00 & 99.95 & 99.98 \\
train\_office\_test\_vacuum\_default  & 97.23 & 94.54 & 97.23 & 95.86 \\
train\_vacuum\_test\_office\_default  & 98.05 & 96.14 & 98.05 & 97.09 \\
\hline
\end{tabular}
\caption{Performance metrics for different configurations.}
\label{table:appendix_ann}
\end{table}


\chapter{Design Artifacts}

In this section, we present the design artifacts used to develop GhostPeerShare. These artifacts serve as a visual aid to represent the author's intent. This is not a comprehensive collection of all classes within GhostPeerShare but only a subset to demonstrate the application's structure and organization.

To optimize the application, Figure \ref{fig:appendix_manifestClass}, a singleton pattern used to manage imported videos, pre-processed byte-count arrays, and imported archives. This system manages the read-and-write flow of data throughout the application. This system enables preprocessing to only be run once and load previously imported videos.

To implement modular support of various media types, GhostPeerShare defines additional media types. Figure \ref{fig:appendix_videoClass} demonstrates the abstraction pattern to support future media types rather than tightly coupling the application to only support videos. Furthermore, Figure \ref{fig:appendix_videoImportDataFlow} demonstrates the logic of adding a trimmed video to the manifest, and Figure \ref{fig:appendix_videoPreprocessDataFlow} is a separate intent to preprocess the video and cache the resulting byte-count array.

Finally, to facilitate the import and export of encrypted data between instances, we used archives to compress and share larger, serialized objects, shown in Figure \ref{fig:appendix_archiveClass}. 
The contents of the \textit{CiphertextVideoZip} and \textit{ImportCiphertextVideoZip} are shown in Figure \ref{fig:ciphertext-archive-contents}.
To convey the intent, regardless of how data is transmitted between devices, the behavior of handling archives are shown in Figure \ref{fig:appendix_archiveDataFlow}.

\begin{figure*}
  \centering
  \includegraphics[width=\textwidth,height=0.8\textheight,keepaspectratio]{Appendix/Figures/manifestClass.png}
  \caption{UML Manifest Class Diagram}
  \label{fig:appendix_manifestClass}
\end{figure*}

\begin{figure*}
  \centering
  \includegraphics[width=\textwidth,height=0.9\textheight,keepaspectratio]{Appendix/Figures/videoClass.png}
  \caption{UML Video Class Diagram}
  \label{fig:appendix_videoClass}
\end{figure*}

\begin{figure*}
  \centering
  \includegraphics[width=\textwidth,height=\textheight,keepaspectratio]{Appendix/Figures/videoImport.png}
  \caption{Video Import Data Flow Diagram}
  \label{fig:appendix_videoImportDataFlow}
\end{figure*}

\begin{figure*}
  \centering
  \includegraphics[width=\textwidth,height=\textheight,keepaspectratio]{Appendix/Figures/videoPreprocess.png}
  \caption{Video Preprocess Data Flow Diagram}
  \label{fig:appendix_videoPreprocessDataFlow}
\end{figure*}

\begin{figure*}
  \centering
  \includegraphics[width=\textwidth,height=\textheight,keepaspectratio]{Appendix/Figures/shareArchiveClass.png}
  \caption{UML Archive Class Diagram}
  \label{fig:appendix_archiveClass}
\end{figure*}

\begin{figure*}
  \centering
  \includegraphics[width=\textwidth,height=\textheight,keepaspectratio]{Appendix/Figures/shareArchiveSequence.png}
  \caption{Archive Data Flow}
  \label{fig:appendix_archiveDataFlow}
\end{figure*}

\end{document}
