%  ========================================================================
%  Copyright (c) 1985 The University of Washington
%
%  Licensed under the Apache License, Version 2.0 (the "License");
%  you may not use this file except in compliance with the License.
%  You may obtain a copy of the License at
%
%      http://www.apache.org/licenses/LICENSE-2.0
%
%  Unless required by applicable law or agreed to in writing, software
%  distributed under the License is distributed on an "AS IS" BASIS,
%  WITHOUT WARRANTIES OR CONDITIONS OF ANY KIND, either express or implied.
%  See the License for the specific language governing permissions and
%  limitations under the License.
%  ========================================================================
%

% Documentation for University of Washington thesis LaTeX document class
% by Jim Fox
% fox@washington.edu
%
%    Revised 2020/02/24, added \caption()[]{} option.  No ToC.
%
%    Revised for version 2015/03/03 of uwthesis.cls
%    Revised, 2016/11/22, for cleanup of sample copyright and title pages
%
%    This document is contained in a single file ONLY because
%    I wanted to be able to distribute it easily.  A real thesis ought
%    to be contained on many files (e.g., one for each chapter, at least).
%
%    To help you identify the files and sections in this large file
%    I use the string '==========' to identify new files.
%
%    To help you ignore the unusual things I do with this sample document
%    I try to use the notation
%       
%    % --- sample stuff only -----
%    special stuff for my document, but you don't need it in your thesis
%    % --- end-of-sample-stuff ---


%    Printed in twoside style now that that's allowed
%
 
\documentclass [11pt, proquest] {uwthesis}[2020/02/24]
 
%
% The following line would print the thesis in a postscript font 

% \usepackage{natbib}
% \def\bibpreamble{\protect\addcontentsline{toc}{chapter}{Bibliography}}

\setcounter{tocdepth}{1}  % Print the chapter and sections to the toc
 

% ==========   Local defs and mods
%

% --- sample stuff only -----
% These format the sample code in this document

\usepackage{alltt}  % 
\newenvironment{demo}
  {\begin{alltt}\leftskip3em
     \def\\{\ttfamily\char`\\}%
     \def\{{\ttfamily\char`\{}%
     \def\}{\ttfamily\char`\}}}
  {\end{alltt}}
 
% metafont font.  If logo not available, use the second form
%
% \font\mffont=logosl10 scaled\magstep1
\let\mffont=\sf
% --- end-of-sample-stuff ---

% Table 5.1
\usepackage{amsmath}
\usepackage{graphicx}
\usepackage{array}

\begin{document}
 
% ==========   Preliminary pages
%
% ( revised 2012 for electronic submission )
%

\prelimpages
 
%
% ----- copyright and title pages
%
\Title{GhostPeerShare}
\Author{Jeffrey Murray Jr}
\Year{2024}
\Program{Cybersecurity Engineering}

\Chair{Brent Lagesse}{Ph.D}{Computing \& Software Systems}
\Signature{Geethapriya Thamilarasu}
\Signature{Yang Peng}

\copyrightpage

\titlepage  

 
%
% ----- signature and quoteslip are gone
%

%
% ----- abstract
%


\setcounter{page}{-1}
\abstract{%
This paper presents advancements in the integration of Fully Homomorphic Encryption (FHE) within mobile applications through the development of the Fully Homomorphic Encryption Library (FHEL) Plugin and the Distance Measure Plugin for Flutter. By leveraging Microsoft SEAL, we demonstrate high performance and responsiveness for encrypted computations on mobile devices, addressing the need for privacy and security in mobile software development. Our results indicate high accuracy and consistency in deriving video similarity scores using Kullback-Leibler Divergence, Bhattacharyya Coefficient, and Cramer’s Distance, facilitating effective secure data exchange. This work opens pathways for future applications of FHE, particularly in critical areas such as medical devices, enabling secure interactions between mobile applications and encrypted data from cloud service providers. Ultimately, our findings highlight the potential of FHE to enhance user privacy and security in modern mobile applications, laying the groundwork for future innovations in this field.
}
 
%
% ----- contents & etc.
%
\tableofcontents
\listoffigures
\listoftables
 
%
% ----- glossary 
%
\chapter*{Glossary}      % starred form omits the `chapter x'
\addcontentsline{toc}{chapter}{Glossary}
\thispagestyle{plain}
%
\begin{glossary}
\item[Fully Homomorphic Encryption (FHE)] A type of encryption that allow computations over encrypted data without access to the decryption key.

\item[Cheon-Kim-Kim-Song (CKKS) Scheme] A fully homomorphic encryption scheme that supports approximate arithmetic on real numbers used for decimal-based computations.

\item[Simple Encrypted Arithmetic Library (SEAL)] An open-source library developed by Microsoft for performing fully homomorphic encryption arithmetic on encrypted data.

\item[Kullback-Leibler Divergence (KLD)] A measure of how one probability distribution diverges from another, ranges from 0 to positive infinity, where 0 represents identical probability distributions.

\item[Bhattacharyya Coefficient (BC)] A similarity measure that quantifies the amount of overlap between two probability distributions, ranges from 0 to 1, where 1 represents identical probability distributions.

\item[Cramer's Distance (CD)] A statistical measure used to quantify dissimilarity between two probability distributions, ranges from 0 to 1, where 0 represents identical probability distributions.

\item[Flutter] An open-source, cross-platform framework by Google for building natively compiled applications for mobile, web, and desktop from a single codebase.

\item[Dart] A programming language optimized for building fast applications on any platform, used in the development of Flutter apps.

\item[FFmpeg] A multimedia framework commonly used for video and audio processing, used here for video preprocessing in mobile applications.

\item[OpenCV] An open-source computer vision and machine learning software library that provides tools for image and video processing.

\item[GitHub Actions] A CI/CD tool that automates tasks such as building, testing, and deploying software, supporting the automation of plugin testing and deployment in this project.

\end{glossary}

 
%
% ----- acknowledgments
%
\acknowledgments{% \vskip2pc
  % {\narrower\noindent
  The author wishes to express sincere appreciation to
  University of Washington, where he has had the opportunity
  to work with the \TeX\ formatting system,
  and to the author of \TeX, Donald Knuth, {\it il miglior fabbro}.
  % \par}
}

%
% end of the preliminary pages

%
% ==========      Text pages
%

\textpages
 
% ========== Chapter 1
 
\chapter {Introduction}
 
\section{Problem Statement}
\label{sec:Problem Statement}
The Microsoft Simple Encrypted Arithmetic Library (SEAL) \cite{sealcrypto} requires significant pre-requisite knowledge to compile and embed the binary within resource-constrained devices. These prerequisites have limited the accessibility of SEAL, posing a significant obstacle to the adoption within the mobile community.

\section{Stakeholders}
Microsoft was a pioneer in the Fully Homomorphic Encryption research, leading the development of the Simple Encryption Arithmetic Library, SEAL in 2018. SEAL established a foundation for efficient homomorphic encryption, making it accessible for developers and researchers. The open-source community soon contributed to the field with OpenFHE, a flexible, community-driven project that was initially released in 2020. OpenFHE expands upon the capabilities of SEAL, offering a collaborative platform for researchers and developers interested in advancing encryption technology.

Flutter, a versatile and state-of-the-art framework by Google, now captures 46\% of the cross-platform framework market (Vailshery 2024). Recognized for its ability to support high-performance, cross-platform applications, Flutter includes a powerful plugin system with a Foreign Function Interface (FFI). This interface enables seamless integration with C libraries, offering null safety, managed memory, and compatibility across major operating systems, including Android, Linux, macOS, iOS, and Windows.

By leveraging Dart, a more portable programming language, developers and security researchers can interact directly with FHE backend libraries. This integration reduces the complexity of working with SEAL or OpenFHE, allowing developers to rapidly build cross-platform applications with homomorphic encryption capabilities. Furthermore, security researchers can extend the application to support additional C/C++ libraries, broadening the scope of FHE implementations and facilitating wider distribution to end-users.

\section{Contributions}
\label{sec:Contributions}
The Pyfhel library \cite{Ibarrondo2021-Pyfhel} provides a native Python interface to SEAL cryptosystems, abstracting core SEAL functionalities into a more generic interface. While Pyfhel supports cross-platform compatibility on Windows, Linux, and macOS, it currently lacks support for mobile platforms, e.g. Android and iOS. This project adopts a similar approach, creating a modular interface that supports multiple backend libraries.

This project contributes a modular, Dart plugin that abstracts and implements Microsoft Simple Encrypted Arithmetic Library. The abstraction design enables the integration and versioning of additional backend libraries. With the use of CMake configurations, each C library can be synchronized and recompiled as new versions become available, simplifying the update process. Automating these compilations reduces maintenance overhead, ensuring that the libraries remain up-to-date with minimal manual intervention.

To assess the library's performance and accuracy, we re-implement the Proof of Presence Share \cite{Lagesse2021-PopShare} methodology. This method calculates similarity scores between videos without exposing their content. We set up cameras from various angles to record simultaneously and compare the videos for similarity, as well as, perform comparisons against other scenes of the same duration. Using a supervised artificial neural network, we classify the distance measures of Kullback-Leibler Divergence \cite{Kullback1951-bg}, Bhattacharyya Coefficient \cite{Bhattacharyya1933-fw}, and Cramer Distance \cite{Cramer1928-sw} for training and testing. The binary classification model is trained to differentiate between pairs of scenes, classifying them as either similar or different based on the unique representations generated by our system. By analyzing the model’s accuracy through many predictions, we gain insight into the precision and reliability of our implementation in distinguishing matching scenes without revealing the videos’ content.


% ========== Chapter 2
 
\chapter{Background}

\section{Flutter}
To support Android and Linux operating systems, Flutter is a modern, open-source framework designed to build natively compiled applications for mobile, web, and desktop using a single codebase. It is based on Dart and uses widgets and hot reload for efficient and fast development. Dart is an object-oriented language with C-style syntax, Dart includes advanced features such as garbage collection for memory management, synchronous and asynchronous handlers, null safety, and compile-time data type checking.

We selected Flutter for its low development cost and Foreign Function Interface (FFI), a Dart plugin that allows the application to invoke functions from pre-compiled C libraries. Flutter and Dart are unique, in that, there are no direct alternatives. Rather, there is a collection of languages and frameworks to achieve a similar implementation, such as React Native and JavaScript or Kotlin or Swift. Each language has its own distinct compilation methods. Therefore, Flutter and Dart are a more suitable choice for this project.

\section{Simple Encrypted Arithmetic Library}

To perform computations on encrypted data, this project utilizes Simple Encrypted Arithmetic Library (SEAL) \cite{sealcrypto}, an open-source homomorphic encryption library developed by Microsoft. The library supports limited arithmetic including addition, subtraction, and multiplication to be performed directly on encrypted data without needing to decrypt it. We chose SEAL primarily for its maturity, prevalent adoption in research, and use in Proof of Presence Share \cite{Lagesse2021-PopShare}.

SEAL supports three Fully Homomorphic Encryption (FHE) schemes: Brakerski-Fan-Vercauteren (BFV) \cite{fan2012-bfv}, Brakerski-Gentry-Vaikuntanathan (BGV) \cite{brakerski2012-bgv}, and Cheon-Kim-Kim-Song (CKKS) \cite{Cheon2017-CKKS}. BFV and BGV are integer-based FHE schemes that allow for computations on encrypted integers, but differ in their ciphertext structure and noise management techniques. BFV encodes the message with the most significant bits of the ciphertext and accumulates more noise growth for each multiplication circuit. When too much noise accumulates, the ciphertext cannot be decrypted. For simpler computations, BFV offers better performance. BGV encodes the message in the least significant bits and manages noise through modulus switching. This difference in noise management makes BGV suitable for deeper computations with many sequential operations. 

CKKS is specifically designed for scenarios where approximate arithmetic is acceptable, and this trade-off allows for more efficient computations on encrypted real numbers. This scheme is efficient in performing computations, by representing real numbers as complex numbers. The encoding allows for efficient arithmetic operations, specifically addition, multiplication, and subtraction. When decoding, the result is transformed into a meaningful representation of the true value, with some degree of error. 

\section{Distance Measures}

To accurately measure the similarity between videos while preserving privacy, this project utilizes Kullback-Leibler Divergence (KLD), Bhattacharyya Coefficient (BC), and Cramer’s Distance (CD). KLD (Kullback and Leibler 1951) is a non-symmetric measure of the difference between two probability distributions. BC (Bhattacharyya 1946) is a symmetric measure of the overlap between two probability distributions. CD (Cramér 1928) is a symmetric measure of the difference between two probability distributions. Following the work of Pop-Share (Lagesse et al. 2021), which established these measures as effective baselines for video similarity analysis. These three metrics were chosen over other probability distribution algorithms due to their reliance on basic arithmetic for homomorphic computations.


% ========== Chapter 3
 
\chapter{Related Work}

In this chapter, we provide an overview of the current research landscape surrounding Fully Homomorphic Encryption, relevant Video Similarity algorithms, and Proof of Presence Share, which serves as the baseline for our implementation.

\section{Fully Homomorphic Encryption Libraries}

Open-Source Fully Homomorphic Encryption Library, OpenFHE [10], is a state-of-the-art C++ library born from a merger of PALISADE, HElib and HEAAN. OpenFHE supports modern schemes and can be configured to use hardware acceleration.

From existing systems, OpenFHE adapted the modular design of PALISADE [3] with the built-in support of BGV, BFV, CKKS and FHEW schemes. From HElib [2] an efficient homomorphic encryption library primarily focused on BGV and CKKS schemes, pioneered research and early development in this domain. HEAAN [11] was integrated to support arithmetic operations on approximate numbers while other schemes only support integer based arithmetic.

Compared to SEAL, OpenFHE library has an actively growing community with modular integrations of existing systems. The documentation and development guides for OpenFHE are much more modern and comprehensive. However, SEAL is much more established within the research community. Regarding performance, there have not been any peer reviewed publications directly comparing the performance of Microsoft SEAL and OpenFHE.

\section{Proof of Presence Share}
\label{sec:Related Pop-Share}
Proof of Presence Share (Pop-Share) \cite{Lagesse2021-PopShare} introduces a novel approach for accurately detecting similar video scenes with strong privacy guarantees, presenting two applications of fully homomorphic encryption deployed within a client-server architecture and a distributed peer-to-peer network. To accurately detect similar video scenes, Pop-Share proposed a new application and evaluation of the Similarity of Simultaneous Observation (SSO) \cite{Wu2019-SSO}. 

Developed by Wu and Lagesse, SSO was intended to detect streaming Wi-Fi cameras by comparing the probability distribution between network traffic and recording videos on the network. In addition, SSO contributed a computationally efficient way to transform video frames into probability distribution form to be compared by various statistical measures including Jensen-Shannon Divergence (JSD), Kullback-Leibler Divergence, and Dynamic Time Warping (DTW) \cite{Sakoe1978-dtw}. The pre-processing approach of SSO slices the video into one-second segments, computes the total number of the bytes for all frames in each segment, and normalizes the byte count array so that it summates to one. However, JSD and DTW could not be simplified enough to be used in fully homomorphic operations, so Pop-Share leveraged Cramer Distance and Bhattacharyya Coefficient to handle the encrypted similarity score measures.

Pop-Share was developed in 2020, its implementation depended on SEAL 3.3 to handle fully homomorphic operations. Once both videos were preprocessed, the transformation was irreversible, having strong privacy guarantees. In addition, the arrays were encrypted and computed using the plaintext counterpart, resulting in less noise accumulation than when using only ciphertext. In a distributed peer-to-peer application, the initiator shares an encrypted video for the recipient to apply their plaintext array onto the untrustworthy ciphertext. The modified ciphertext was returned to the originator to be decrypted and the summation of the decrypted array represents the corresponding similarity score. This was performed for each similarity score measure. A notable limitation of this approach is that their implementation was tightly coupled with a static version of SEAL, making the results difficult to reproduce.

This method demonstrated that SSO generates a unique, one-way signature of videos that can be used to compare for similarity without access to the video's content. For this project, the results from Pop-Share serve as a baseline comparator for the modular Flutter re-implementation of the Pop-Share peer-to-peer application.

\section{Video Similarity}

As a part of SSO and Pop-Share, a video was represented in probability distribution form, in order to compare two probability distributions, there are many methods to produce a similarity score. In this section, we cover all considerations for generating a similarity score, as well as, acknowledge alternative approaches to comparing videos.

Cramer’s Distance (CD) [15] is a measure used to quantify the dissimilarity between two probability distributions. It is based on the concept of the characteristic function and finds applications in statistical inference and hypothesis testing. One limitation of Cramer’s Distance is its sensitivity to sample size, which can result in inaccuracies, particularly when the sample size is small.
Bhattacharyya Coefficient (BC) [16] is a measure that quantifies the amount of overlap between two probability distributions, making it useful in various fields such as pattern recognition and image processing. The coefficient ranges from 0 to 1, where a value closer to 1 indicates higher similarity between the distributions.

Kullback-Leibler Divergence (KLD) [17] is a widely used measure of how one probability distribution diverges from a second, expected probability distribution. A limitation of KLD is that it is not symmetric, meaning that the divergence from distribution P to Q is not necessarily the same as from Q to P, which may complicate its interpretation.
Jensen-Shannon Divergence (JSD) [18] is a symmetric version of KLD that quantifies the similarity between two probability distributions. It is widely used in information retrieval and clustering. A limitation is that while it provides a measure of similarity, it may not capture finer details of the distributions being compared.

The Pearson Correlation Coefficient (PCC) [19] measures the linear correlation between two variables, producing a value between -1 and 1. It is a fundamental statistical tool in various fields, including psychology and economics. However, this coefficient assumes a linear relationship and may not capture non-linear associations, which could lead to misleading conclusions in certain contexts. 
Dynamic Time Warping (DTW) [20] is an algorithm used to measure similarity between two temporal sequences that may vary in speed. It has applications in speech recognition, data mining, and video analysis. One limitation of DTW is its computational complexity, particularly for long sequences, which can make it less suitable for real-time applications. 
Alternatives to SSO are advanced applications on top of existing similarity score algorithms. These novel approaches are typically not restricted to resource-constrained environments.

Using content-based algorithms, Shan and Lee [21] presented a series of optimal mapping algorithms to detect similarity between video frames of unequal sizes. In addition, Wu, Zhuang, and Pan [22] present a similar shot graphing algorithm with the intention of identifying similar videos within a large database of videos. An enhancement of Bhattacharyya, was proposed by Loza, Mihaylova, Canagarajah, and Bull [23] to generate a particle filter that would iterate over each sample to predict and resample based on the similarity. Overall, these methods may be adapted to be implemented with fully homomorphic encryption, however they would need to be simplified to use basic arithmetic, and may require many parameters to produce a score.


% ========== Chapter 4
 
\chapter{Design}

GhostPeerShare is comprised of three major components, Fully Homomorphic Encryption Library, written as a Dart plugin, Similarity Score dart plugin, and Video Similarity Flutter application. In this chapter, we provide an in-depth walk through of each component.

\section{Fully Homomorphic Encryption Library}
\label{sec:Fully Homomorphic Encryption Library}
This implementation adheres to the core design principle of modularity, remaining agnostic to the underlying C++ library. Instead of directly invoking methods within the plugin, the plugin uses the Bridge and Adapter structural design patterns inspired by the work of Alberto Ibarrondo and Alexander Viand \cite{Ibarrondo2021-Pyfhel}. The Bridge pattern decouples the high-level Dart API from the specific C++ implementation by defining an abstract interface. The Adapter pattern connects the Dart interface to the C++ implementation through a C-based intermediary, translating Dart requests into operations understood by the native library. This approach enables compatibility with multiple Fully Homomorphic Encryption (FHE) backends without modifying the Dart API.

The Fully Homomorphic Encryption Library provides a modular API that integrates Microsoft SEAL with the Flutter ecosystem. It offers a straightforward interface for developers, abstracting low-level complexities while delivering access to advanced encryption functionalities required for secure applications. The topology of this Dart plugin ensures a clear separation of responsibilities to promote maintainability and extensibility. Table \ref{table:class-structure} outlines the distinct roles of each component, including the high-level API, the adapter interface, and the concrete implementation.

\begin{table}[t]
\caption{Class Structure Overview}
\centering
\begin{tabular}{|l|l|p{9cm}|}
\hline
\textbf{File} & \textbf{Class} & \textbf{Description} \\ \hline
seal.dart    & Seal    & Entry-point for the end-user API in Dart \\ \hline
afhe.dart    & Afhe    & Dart adapter connecting to the C interface \\ \hline
fhe.cpp      & N/A     & C interface bridging Dart and C++ \\ \hline
afhe.h       & Afhe    & Pure abstract class defining the FHE contract \\ \hline
aseal.cpp    & Aseal   & Concrete implementation of the Afhe abstraction \\ \hline
\end{tabular}
\label{table:class-structure}
\end{table}


Each file plays a specific role in this architecture. The \textit{seal.dart} file provides the primary entry point for the end-user API, allowing developers to interact with the encryption library in Dart. The \textit{afhe.dart} file acts as the adapter, invoking lower-level C functions. The Afhe class exposes abstracted data types such as Keys, Ciphertext, and Plaintext objects. The \textit{fhe.cpp} file defines the C interface. It exposes the inherited methods from \textit{afhe.h} and references the memory address of the underlying C++ object. Depending on the reference, the corresponding concrete implementation will be invoked. For example, \textit{aseal.cpp} manages Microsoft SEAL objects.

The Foreign Function Interface (FFI) is this plugin's core dependency; It enables Dart to interact with the pre-compiled C binary. For each target platform, the dynamically linked library must be accessible on the local file system. This plugin exposes methods to cast native C data types into Dart objects and vice versa. Our implementation creates Dart objects that mirror their corresponding underlying C++ class. For example, in Microsoft SEAL, a Ciphertext can be saved or loaded. We expose \textit{save\_ciphertext} and \textit{load\_ciphertext} in the C interface. In Dart, we create a Ciphertext class that contains both \textit{save} and \textit{load}, referencing the memory address of the underlying abstract Ciphertext object. This implementation pattern is consistent and transparent for security researchers familiar with the underlying C++ API. For new developers, our implementation mirrors the structure of Microsoft SEAL, providing a clear and familiar interface.

In order to prevent breaking changes, unit tests ensure that the functional behavior of the library remains consistent and reliable throughout the development and release. Unit tests serve as a safety net. They ensure that existing features function as expected when new changes occur. This process reduces the risks of regressions. For SEAL, the authors developed a set of examples that walk the user through their APIs. As a part of the unit tests, we re-implemented these examples with our interface to increase our confidence that our implementation did not introduce any new or unexpected behaviors. For automation, GitHub Actions compiles and executes unit tests using GoogleTest \cite{GoogleTest} with CMake. In total, There are 78 unit tests: 35 in C and 43 in Dart.

In order to distribute this plugin to Linux and multiple Android flavors, GitHub Actions employs Conan \cite{Conan}. Conan automates the management of dependencies and packaging, specifically addressing the versioning of Microsoft SEAL and other backend libraries. For Linux, Conan streamlines the building and packaging process by automatically resolving dependencies tailored to various distributions (e.g., Ubuntu, Fedora, and CentOS) while ensuring compatibility with different versions of Microsoft SEAL and system architectures (e.g., x86, ARM). Similarly, Conan simplifies cross-compilation for Android by allowing developers to specify configurations for various architectures and API levels, ensuring that the plugin can be compiled for the supported CPU architectures x86\_64, ARMv8, and ARMv7. The compiled binaries are all hosted on all platforms and are available for download for each release of the Dart plugin. When developers add the plugin to their dependencies, the binaries are pulled down from GitHub and stored within their local, versioned dependency cache. This automated approach enhances the consistency and reliability of the plugin across different environments, facilitating smoother deployment and easier updates for users.
\input{4 Design/4.2 Similarity Score Plugin}
\section{Video Similarity Application}
To perform computations on encrypted floating-point numbers, we utilize the CKKS scheme [11] with a polynomial modulus degree of 4096. CKKS is a homomorphic encryption scheme to perform computations of approximate arithmetic on real numbers. This scheme offers out-of-the-box support for decimal value computations compared to other schemes like BFV and BGV, which primarily focus on integer arithmetic. The choice of 4096 as the polynomial modulus degree balances the need for strong security and accuracy with practical computational performance. While a larger degree enhances security and allows for more operations before result degradation, with the trade off of increasing computation time and ciphertext size.

In order to detect similarities between two videos, we represent each video as a probability distribution of byte size changes between one-second segments, shown in Figure \ref{fig:preprocess-data-flow}. To transform raw video into comparable probability distributions, we count the number of bytes in each frame, calculate the sum of frame lengths in each segment, and then normalize the array of segments between zero and one, known as a Probability Distribution Form (PDF), ensuring that the distributions for both videos have the same scale and can be directly compared. This approach is borrowed from the Similarity of Simultaneous Observation [12], and provides strong privacy guarantees, as if the encryption is broken, the raw byte data cannot be reconstructed from the normalized byte array due to the loss of information during the aggregation and normalization process. We have considered a motion estimation approach with the Lucas-Kanade method [23] supported by OpenCV, which could provide more detailed information about video content. However, for this study, it is crucial to compare our approach to the established baseline metrics, and new methods would hinder this comparison.

\begin{figure}[ht]
    \centering
    \includegraphics[width=\textwidth]{4 Design/4.3 Preprocess Data Flow.png}
    \caption{Pre-process Data Flow}
    \label{fig:preprocess-data-flow}
\end{figure}

To compute multiple similarity scores using homomorphic encryption, separate byte arrays are encrypted for each score. The recipient can perform computations with plaintext decimal values against the corresponding encrypted element in the array. This approach leverages the unique ability of fully homomorphic encryption and was borrowed from Pop-Share, as this approach is much faster than comparing ciphertext against ciphertext. 
For Kullback-Leibler Divergence, we share the PDF and the natural log of the PDF. For each encrypted double in the PDF, we subtract the natural log of the plaintext element and multiply the difference. The product remains encrypted and returned to the originator, where it can be decrypted and the summation is the similarity score.
For Bhattacharyya Coefficent, we share the square root of the PDF. For each encrypted double, we multiply the ciphertext by the square root of the plaintext element. The product remains encrypted and returned to the originator, where it can be decrypted and the summation is the similarity score.
For Cramer Distance, we share Cumulative Distribution Form (CDF) from PDF. The CDF is the cumulative sum of all of the elements in the PDF, such that the last value of CDF is equal to 1. For each encrypted double, we square the difference of the plaintext element. The product remains encrypted and returned to the originator, where it can be decrypted and the square root of the summation is the similarity score.


% ========== Chapter 5

\chapter{Results}

In this section, we compare our implementation to the foundational contributions of SSO and PPVS. In order to evaluate the accuracy of Fully Homomorphic Encryption Library plugin in Dart, we measure the amount of noise generated during encrypted operations. Next, we compare our implementation of our distance measures using the original pre-processed video frames from SSO. Next, we compare the same video from various angles and determine the accuracy of our implementation. Next, we compare the distance measures for similar scenes against different scenes. Finally, we compare the performance across linux and android devices.

\section{Noise Accumulation}

When performing fully homomorphic operations, a small amount of noise is introduced that affects the accuracy of the decrypted distance measure. If too much noise is accumulated, the plaintext cannot be recovered. Table \ref{table:mean-error} represents the baseline Mean Error from Pop-Share \cite{Lagesse2021-PopShare}, compared to our implementation. For Kullback-Leibler Divergence, Cramer Distance, and Bhattacharyya Coefficient our implementation introduced half the amount of noise than the original.

As illustrated in Figure \ref{fig:pop-share-distance-measure}, our Dart implementation exhibits less variance compared to the Python baseline, allowing for a clearer distinction in mean error scores. This reduced variance indicates a more stable performance in our implementation, reinforcing the reliability of the distance measures employed. Furthermore, through automated unit tests, we have ensured that the scores do not differ more than $1.0e-7$ epsilon, confirming the consistency and accuracy of our calculations.

\begin{table}[t]
\caption{Difference of Mean Error for each algorithm}
\centering
\begin{tabular}{|c|c|c|c|}
\hline
\textbf{Function} & \textbf{Pop-Share ($PS$)} & \textbf{GhostPeerShare ($GPS$)} & \textbf{Difference ($PS-GPS$)} \\
\hline
KLD     & $4 \times 10^{-9}$ & $1.71 \times 10^{-11}$ & $3.98 \times 10^{-9}$ \\
Cramer  & $4 \times 10^{-4}$ & $1.61 \times 10^{-9}$ & $3.99 \times 10^{-4}$ \\
BC      & $8 \times 10^{-4}$ & $3.85 \times 10^{-10}$ & $7.99 \times 10^{-4}$ \\
\hline
\end{tabular}
\label{table:mean-error}
\end{table}


Finally, the parameter choices within the encryption scheme-such as polynomial modulus degree, encoder scalar, or array of modulus sizes-also play a significant role in noise management.



% ========== Chapter 6

\chapter{Discussion}

\input{6 Discussion/6.1 Limitations}
\input{6 Discussion/6.2 Qualitative Analysis}

%
% ==========   Bibliography
%
\nocite{*}   % include everything in the uwthesis.bib file
\bibliographystyle{plain}
\bibliography{uwthesis}
%
% ==========   Appendices
%
\appendix
\raggedbottom\sloppy
 
% ========== Appendix A
 
\chapter{Where to find the files}
 
The uwthesis class file, {\tt uwthesis.cls}, contains the parameter settings,
macro definitions, and other \TeX nical commands which
allow \LaTeX\ to format a thesis.  
The source to
the document you are reading, {\tt uwthesis.tex},
contains many formatting examples
which you may find useful.
The bibliography database, {\tt uwthesis.bib}, contains instructions
to BibTeX to create and format the bibliography.
You can find the latest of these files on:

\begin{itemize}
\item My page.
\begin{description}
\item[] \verb%https://staff.washington.edu/fox/tex/thesis.shtml%
\end{description}

\item CTAN
\begin{description}
\item[]  \verb%http://tug.ctan.org/tex-archive/macros/latex/contrib/uwthesis/%
\item[]  (not always as up-to-date as my site)
\end{description}

\end{itemize}

\vita{Jim Fox is a Software Engineer with IT Infrastructure Division at the University of Washington.
His duties do not include maintaining this package.  That is rather
an avocation which he enjoys as time and circumstance allow.

He welcomes your comments to {\tt fox@uw.edu}.
}


\end{document}
